\documentclass{article}
\title{Informatik \rotatebox[origin=c]{180}{D}\raisebox{2pt}{:} -- Blatt 6}
\author{Rasmus Diederichsen}
\date{\today}
\usepackage[utf8]{inputenc}
\usepackage[ngerman]{babel}
\usepackage[table,svgnames]{xcolor}
\usepackage{wasysym}
\let\iiint\relax % wasysym-amsmath interference
\let\iint\relax
\usepackage{microtype,
   booktabs,
   multirow,
   listings,
   cwpuzzle,
   tikz,
   multicol,
   IEEEtrantools,
   array,
   amsmath,
   amssymb,
   graphicx, 
lmodern}
\usepackage[pdftitle={Informatik D -- Blatt 6}, 
   pdfauthor={Rasmus Diederichsen}, 
   hyperfootnotes=true,
   colorlinks,
   bookmarksnumbered = true,
   linkcolor = blue,
   plainpages = false,
citecolor = blue]{hyperref}
\usepackage[T1]{fontenc}
\usepackage[all]{hypcap}
\renewcommand{\thesection}{}
\renewcommand{\thesubsection}{Aufgabe \arabic{section}.\arabic{subsection}}
\renewcommand{\thesubsubsection}{}
\setcounter{section}{5}

\begin{document}

\maketitle

\section{} 
\subsection{} 

\usetikzlibrary{arrows,automata,positioning}
\tikzset{every node/.append style={font=\sf}}

Es sei $\Gamma^\prime = \Gamma\cup\{\$\}$.

\hspace{-1cm}
   \begin{tikzpicture}[auto,thick,>=stealth',
         initial text={\sffamily start},node distance=1cm and 2cm]
      \node[state,initial] (Zs) {$Z_s$};
      \node[state] (Z1) [right=1.2cm of Zs]  {$Z_1$}; 
      \node[state] (Z2) [right=of Z1]  {$Z_2$};
      \node[state] (Z3) [right=of Z2]  {$Z_3$};
      \node[state] (Z4) [above=2.5cm of Z2] {$Z_4$};
      \node[state] (Z5) [above right=.6cm and .6cm of Z3] {$Z_5$};
      \node[state] (Z6) [right=of Z5] {$Z_6$};
      \node[state] (Z8) [below right=.6cm and .6cm of Z3] {$Z_8$};
      \node[state] (Z7) [right=of Z8] {$Z_7$};
      \node[state] (Z9) [below left=1.2cm and .6cm of Z3,draw=none] {};
      \node[state] (Z10) [below left=.1cm and 1.5cm of Z9] {$Z_{10}$};
      \node[state] (Z11) [below=of Z1] {$Z_{11}$};

      \path[->]
      (Zs) edge node {$\varepsilon,\$/\#\$$} (Z1)
      (Z1) edge node {$a,\star\in\Gamma/b\star$} (Z2)
      (Z1) edge[bend left=60,looseness=1] node {$c,a/ba$} (Z3)
      (Z2) edge[bend left] node {$a,b/\varepsilon$} (Z3)
      (Z2) edge[bend left=50] node[rotate=70,pos=.9,xshift=.3cm,yshift=.2cm] {$\varepsilon,\star\in\Gamma^\prime/\varepsilon$} (Z4)
      (Z2) edge[bend right] node {$c,b/cc$} (Z3)
      (Z3) edge[bend left,pos=.8] node {$c,a/cc$} (Z5)
      (Z4) edge[loop above] node {$\varepsilon,\star\in\Gamma^\prime/\varepsilon$} ()
      (Z5) edge[bend left] node {$\varepsilon,c/cc$} (Z6)
      (Z6) edge[bend left] node {$\varepsilon,c/ac$} (Z7)
      (Z7) edge[bend left] node {$\varepsilon,a/ba$} (Z8)
      (Z8) edge[bend left] node[right,outer sep=8pt,pos=1] {$\varepsilon,b/ab$} (Z3)
      (Z3) edge[bend left=60,looseness=.7] node[above] {$a,\star\in\Gamma/\varepsilon$} (Z1)
      (Z3) edge[bend left] node {$\circ\in\Sigma,\circ/\circ\circ$} (Z10)
      % (Z9) edge[bend left] node {$\varepsilon,\circ/\circ\circ$} (Z10)
      (Z10) edge[bend left] node {$\varepsilon,\circ/\circ\circ$} (Z11)
      (Z11) edge[bend left] node {$\varepsilon,\circ/\circ\circ$} (Z1);
\end{tikzpicture}

\subsection{} 

\subsubsection{Schritt 1}

\newcommand{\R}[1]{R_{(#1)}}
\renewcommand{\r}{\rightarrow}
\renewcommand{\arraystretch}{1.5}

\begin{center}
   \textcolor{teal}{$\forall Z \in \mathcal{Z} : S \rightarrow R_{(Z_{start},\#,Z)}$}
\end{center}

Es resultieren die Regeln
\begin{center}
   \fbox{
\begin{tabular}{>{$}l<{$}>{$}l<{$}}
   S \rightarrow R_{(Z_1,\#,Z_1)} & S \rightarrow R_{(Z_1,\#,Z_2)}
\end{tabular}}
\end{center}

\subsubsection{Schritt 2}

\begin{center}
\tikzset{auto,thick,>=stealth',initial text={\sffamily start},node distance=1cm and 2cm}
\textcolor{teal}{$\forall Z_0$, \raisebox{-1em}{\tikz{\node[state] (Z1)
      {$Z_1$};\node[state] (Z2) [right=of Z1] {$Z_2$};
\path[->] (Z1) edge node {$a,b/c$} (Z2);}} : $R_{(Z_1,b,Z_0)} \rightarrow aR_{(Z_2,c,Z_0)}$}
\end{center}

Es resultieren die Regeln
\begin{center}
   \fbox{
      \begin{tabular}{>{$}l<{$}>{$}l<{$}}
         \R{Z_1,b,Z_1} \r b\R{Z_1,b,Z_1} & \R{Z_1,b,Z_2} \r b\R{Z_1,b,Z_2} \\
         \R{Z_2,b,Z_1} \r c\R{Z_2,b,Z_1} & \R{Z_2,b,Z_2} \r c\R{Z_2,b,Z_2} \\
         \R{Z_2,b,Z_1} \r b\R{Z_1,c,Z_1} & \R{Z_2,b,Z_2} \r b\R{Z_1,c,Z_2}
   \end{tabular}}
\end{center}

\subsubsection{Schritt 3}

\tikzset{node distance=2cm} % dafuq do I have to adjust this again?
\begin{center}
   \textcolor{teal}{$\forall$ Übergänge \raisebox{-1em}{\tikz{\node[state] (Z1)
      {$Z_1$};\node[state] (Z2) [right=of Z1] {$Z_2$};
\path[->] (Z1) edge node[above] {$a,b/\varepsilon$} (Z2);}} : $\R{Z_1,b,Z_2} \r a$}
\end{center}

Es resultieren die Regeln

\begin{center}
   \fbox{
      \begin{tabular}{>{$}l<{$}>{$}l<{$}}
         \R{Z_1,b,Z_1} \r b & \R{Z_1,\#,Z_1} \r b \\
         \R{Z_1,c,Z_1} \r c
   \end{tabular}}
\end{center}

\subsubsection{} 

\begin{center}
   \textcolor{teal}{$\forall Z,Z^\prime$, Übergänge \raisebox{-1em}{\tikz{\node[state] (Z1)
      {$Z_1$};\node[state] (Z2) [right=of Z1] {$Z_2$};
\path[->] (Z1) edge node[above] {$a,b/cd$} (Z2);}} : $\R{Z_1,b,Z} \r
a\R{Z_2,c,Z^\prime}\R{Z^\prime,d,Z}$}
\end{center}

Gemäß $\forall X,Y: \R{Z_1,\#,X} \r a\R{Z_2,b,Y}\R{Y,\#,X}$ (es gibt im Graphen
nur einen solchen Übergang) resultieren die Regeln

\begin{center}
   \fbox{
      \begin{tabular}{>{$}l<{$}}
         \R{Z_1,\#,Z_1} \r a\R{Z_2,b,Z_1}\R{Z_1,\#,Z_1}\\
         \R{Z_1,\#,Z_1} \r a\R{Z_2,b,Z_2}\R{Z_2,\#,Z_1} \\
         \R{Z_1,\#,Z_2} \r a\R{Z_2,b,Z_1}\R{Z_1,\#,Z_2} \\
         \R{Z_1,\#,Z_2} \r a\R{Z_2,b,Z_2}\R{Z_2,\#,Z_2} 
   \end{tabular}}
\end{center}

Der gesamte Regelsatz ist also 

\newcommand{\productive}{\cellcolor{teal!10}}
\begin{center}
   \fbox{
      \begin{tabular}{>{$}l<{$}>{$}l<{$}}
         \productive S \rightarrow R_{(Z_1,\#,Z_1)} & S \rightarrow R_{(Z_1,\#,Z_2)}\\
         \cellcolor{lightgray!30} \R{Z_1,b,Z_1} \r b\R{Z_1,b,Z_1} & \R{Z_1,b,Z_2} \r b\R{Z_1,b,Z_2} \\
         \productive \R{Z_2,b,Z_1} \r c\R{Z_2,b,Z_1} & \R{Z_2,b,Z_2} \r c\R{Z_2,b,Z_2} \\
         \productive \R{Z_2,b,Z_1} \r b\R{Z_1,c,Z_1} & \R{Z_2,b,Z_2} \r b\R{Z_1,c,Z_2}\\
         \cellcolor{lightgray!30} \R{Z_1,b,Z_1} \r b & \productive \R{Z_1,\#,Z_1} \r b \\
         \productive \R{Z_1,c,Z_1} \r c \\
         \productive \R{Z_1,\#,Z_1} \r a\R{Z_2,b,Z_1}\R{Z_1,\#,Z_1}\\
         \R{Z_1,\#,Z_1} \r a\R{Z_2,b,Z_2}\R{Z_2,\#,Z_1} \\
         \R{Z_1,\#,Z_2} \r a\R{Z_2,b,Z_1}\R{Z_1,\#,Z_2} \\
         \R{Z_1,\#,Z_2} \r a\R{Z_2,b,Z_2}\R{Z_2,\#,Z_2} \\
   \end{tabular}}
\end{center}

Regeln, die tatsächlich etwas produzieren, sind aquamarin markiert. Regeln, die
etwas produzieren könnten, von der Startvariable aus aber nicht erreichbar sind,
sind grau hinterlegt. Wir kürzen ab

\begin{center}
   \fbox{
      \begin{tabular}{>{$}l<{$}}
         R_1 := \R{Z_1,\#,Z_1} \\
         R_2 := \R{Z_1,c,Z_1} \\
         R_3 := \R{Z_1,b,Z_1} \\
         R_4 := \R{Z_2,b,Z_1}
   \end{tabular}}
\end{center}

und destillieren hieraus

\begin{center}
   \fbox{
      \begin{tabular}{>{$}l<{$}}
         S \r R_1 \\
         R_1 \r b \mid aR_4R_1 \\
         R_2 \r c \\
         R_4 \r cR_4 \mid bR_2
   \end{tabular}}
\end{center}

und notieren, dass $R_3$ nicht erreichbar ist.

\subsection{} 

\lstset{
   basicstyle=\footnotesize\ttfamily,
   breaklines=true,
   commentstyle=\color{blue},
   keywordstyle=\color{purple}\textbf,
   numberstyle=\tiny\color{gray},
   numbers=left,
   stringstyle=\color{olive},
   frame=single
}

Dieser Algorithmus ist vermutlich nicht ausreichend. Ich weiß auch nicht
wirklich, welche Laufzeit er hat.
% Der (vermutlich falsche) Algorithmus l\"auft in $\mathcal{O}\left(|V|^2\right)$
% da f\"ur jeden Knoten die Suche gestartet wird und in deren Verlauf alle anderen
% Knoten maximal zweimal besucht werden.
\begin{lstlisting}[language=Java]
class DeCycle {
   static void deCycle(Graph g)
   {
      for(Vertex v : g.getVertices()) v.visited = false;
      for(Vertex v : g.getVertices()) traverse(v, new ArrayList<Vertex>());
   }
   static List<Vertex> getTargets(List<Arc>)
   {
      /* ... map Arcs to their targets */
   }
   static void print(Graph g)
   {
      /* output graph */
   }
   static void traverse(Graph g, Vertex v, List<Vertex> path)
   {
      path.add(v);
      if(v.visited) 
      contractCycle(path.sublist(path.indexOf(v), path.length()));
      else
      {
         v.visited = true;
         for (Vertex neighbour : getTargets(g.getOutgoingArcs(v)))
         traverse(g, neighbour, new ArrayList<Vertex>(path));
      }
      v.visited = false; // reset for next search
   }
   static void main(String[] argv)
   {
      Graph g = new RandomGraph(Integer.parseInt(argv[0]));
      DeCycle.deCycle(g);
      DeCycle.print(g);
   }
}
\end{lstlisting}

\subsection{} 

Die Ausgangsgrammatik ist

\subsubsection{$G$}
\begin{align*}
\begin{array}{lll}
   S \r D & C \r eEd & E \r C \\
   S \r dA & C \r B \\
   A \r aS & D \r CdDeB \\
   A \r S & D \r A \\
   A \r E & D \r d \\
   B \r db & E \r B \\
   B \r C & E \r BB
\end{array}
\end{align*}

\textbf{Schritt 1}: Erzeuge Regeln $A_x \r x$, ersetze $a,\ldots,e$ in $G$ durch
$A_a,\ldots,A_e$.

\subsubsection{$G^\prime$}

\begin{align*}
\begin{array}{lll}
   S \r D & C \r A_eEA_d & E \r C \\
   S \r A_dA & C \r B & A_a \r a \\
   A \r A_aS & D \r CA_dDA_eB & A_b \r b \\
   A \r S & D \r A & A_d \r d \\
   A \r E & D \r d & A_e \r e \\
   B \r A_dA_b & E \r B \\
   B \r C & E \r BB
\end{array}
\end{align*}

\textbf{Schritt 2}: Entferne Regeln der Form $V \r V$. Folgende Regeln sind zu
eliminieren:

\begin{align*}
\begin{array}{lll}
   \cellcolor{teal!10} S \r D & C \r A_eEA_d & \cellcolor{teal!10} E \r C \\
   S \r A_dA & \cellcolor{teal!10} C \r B & A_a \r a \\
   A \r A_aS & D \r CA_dDA_eB & A_b \r b \\
   \cellcolor{teal!10} A \r S & \cellcolor{teal!10} D \r A & A_d \r d \\
   \cellcolor{teal!10} A \r E & D \r d & A_e \r e \\
   B \r A_dA_b & \cellcolor{teal!10} E \r B \\
   \cellcolor{teal!10} B \r C & E \r BB
\end{array}
\end{align*}

Wir erstellen den zugeh\"origen Graphen.

\begin{center}
   \begin{tikzpicture}[auto,thick,>=stealth',
         initial text={\sffamily start},node distance=.5cm and 1cm]
      \node[state,initial] (S) {$S$};
      \node[state] (D) [above right=of S]  {$D$};
      \node[state] (A) [below right=of S]  {$A$};
      \node[state] (E) [right=of D]  {$E$};
      \node[state] (B) [right=of A]  {$B$};
      \node[state] (C) [above right=of B]  {$C$};

      \path[->]
      (S) edge[color=red] (D)
      (D) edge[color=red] (A)
      (A) edge[color=red] (S)
      (A) edge (E)
      (E) edge (C)
      (E) edge (B)
      (B) edge[bend left,color=red] (C)
      (C) edge[bend left,color=red] (B);
\end{tikzpicture}
\end{center}
 Es existieren Zyklen $S \r D \r A \r S$ und $B \r C \r B$ bzw. $C \r B \r C$.
 Der kontrahierte Graph ist

\begin{center}
   \begin{tikzpicture}[auto,thick,>=stealth',
         initial text={\sffamily start},node distance=.5cm and 1cm]
      \node[state,initial] (S) {$S$};
      \node[state] (E) [right=of S]  {$E$};
      \node[state] (B) [right=of E]  {$B$};

      \path[->]
      (S) edge (E)
      (E) edge (B);
\end{tikzpicture}
\end{center}
 
Wir eliminieren die Senken beginnend mit $B$ und transformieren so die
Regelmenge $\{S \r E, E \r B\}$.

\textcolor{teal}{$E \r B$} wird zu 
\begin{align*}
   \begin{array}{l}
      E \r A_dA_b \\
      E \r A_eEA_d
\end{array}
\end{align*}

\textcolor{teal}{$S \r E$} wird zu 
\begin{align*}
   \begin{array}{lr}
      S \r BB \\
   S \r A_dA_b &
   \multirow{2}{5cm}{$\left.\raisebox{0pt}[\baselineskip]{}\right\}$ vorherige
Elimination}\\
      S \r A_eEA_d
\end{array}
\end{align*}

Die neue Grammatik ist

\subsubsection{$G^{\prime\prime}$}

\begin{align*}
\begin{array}{l}
   S \r A_dS \mid A_aS \mid BB \mid A_dA_b \mid A_eEA_d \mid BA_dSA_eB \mid d \\
   B \r A_dA_b \mid A_eEA_d \\
   E \r A_dA_b \mid A_eEA_d \mid BB \\
   A_a \r a  \\
   A_b \r b \\
   A_d \r d \\
   A_e \r e
\end{array}
\end{align*}

\textbf{Schritt 3}: K\"urzung zu langer Regeln. Wir f\"uhren f\"ur Regeln $V \r
V_1,\ldots,V_k$ mit $k > 2$ Regeln $V \r V_1U_1, U_1 \r V_2U_2,\ldots,U_k \r
V_{k-1}V_k$ ein und erhalten so

\subsubsection{$G^{\prime\prime\prime}$}

\begin{align*}
\begin{array}{l}
   S \r A_dS \mid A_aS \mid BB \mid A_dA_b \mid A_eU_1 \mid BU_2 \\
   U_1 \r EA_d \\
   U_2 \r A_dU_3 \\
   U_3 \r SU_4 \\
   U_4 \r A_eB \\
   B \r A_dA_b \mid A_eU_1 \\
   E \r A_dA_b \mid A_eU_1 \mid BB \\
   A_a \r a  \\
   A_b \r b \\
   A_d \r d \\
   A_e \r e
\end{array}
\end{align*}

Diese Grammatik ist nun in Chomsky Normal Form.

\subsection{} 

Man nehme an, das Pumping Lemma gilt, $n$ bezeichne die zugehörige
Wortmindestlänge. Sei $z = 1^n21^n21^n$. Es existiert nach Vorraussetzung eine
Zerlegung $uvwxy$ von $z$, sodass $|vx|\ge1,|vwx|\le n$ und $uv^iwx^iy \in L \forall i$

Die Fälle, in denen $v,x$ verschiedene Symbole beinhalten (also sowohl 1en als
auch 2en), braucht man gar nicht zu betrachten, da auf diese Weise Teilwörter der
Form $(1^k2)^i$ o.Ä. entstehen können, was natürlich nicht für alle $i$ in $L$
enthalten ist.

Es gibt vier Möglichkeiten, aus welchen Zeichen $vwx$ bestehen kann.
\begin{enumerate}
   \item $vwx = 1^k, k\le n$. In diesem Fall kann durch Pumpen eine beliebige
      Anzahl 1en erzeugt werden, $uv^iwx^iy = 1^l21^n21^n \text{ für } l \neq n\not\in L$.
   \item $vwx = 1^k2, k\le n-1$. In diesem Fall können durch Pumpen beliebig
   viele 1en erzeugt werden falls $|v| > 0$ und eventuell 2en, falls $|x|>0$. Daher
      $uv^iwx^iy \not\in L$.
   \item $vwx = 1^k21^l, 1\le k+l\le n-1$. In diesem Fall können durch Pumpen beliebig
      viele 1en erzeugt werden, eventuell auch beliebig viele 2en, falls $|v|>1$
      oder $|x|>1$. Daher $uv^iwx^iy \not\in L$, da der letzte Teil $1^n$ nicht
      mitgepumpt wird.
   \item $vwx = 21^k, k\le n-1$. In diesem Fall können durch Pumpen beliebig
      viele 1en erzeugt werden, eventuell 2en, falls $|v|\ge 1$. Mithin
      $uv^iwx^iy \not\in L$.
\end{enumerate}

In keinem Fall ist $uv^iwx^iy \in L$, was einen Widerspruch zur Annahme
darstellt.

\subsection{} 

Bei der Umwandlung eines NDKA-AdLK in eine kontextfreie Grammatik können Regeln der folgenden Arten generiert werden
\begin{enumerate}
	\item $V \rightarrow \Sigma$
  \item $V \rightarrow V$
  \item $V \rightarrow \Sigma \times V$
  \item $V \rightarrow \Sigma \times V \times V$
  \item ($V \rightarrow \varepsilon$)
\end{enumerate}

Dieses Problem ist nur dann lösbar, wenn die Grammatik nicht linksrekursiv ist
(also kein Regeln der Form $A\rightarrow AB$ enthält).

Um die Grammatik in GNF zu überführen, geht man wie folgt vor:
\begin{lstlisting}[mathescape]
entferne Kreise in Regeln V$_i$ -> V$_j$
while exists Regel A -> V$_i$
   forall Regeln V$_i$ -> * 
      kreiere Regel A -> rhs(V$_i$)
   end
   entferne A -> V$_i$
   entferne Regel V$_i$ -> * falls !exists Regel R := V -> V$^*$ mit V$_i$ in rhs(R)
end
while exist Regel A -> V$_1$V$_2$
   forall Regeln V$_1$ -> * kreiere Regel A -> rhs(V$_1$)V$_2$
   entferne Regel A-> V$_1$V$_2$
end
\end{lstlisting}
\end{document}
