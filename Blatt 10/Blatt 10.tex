\documentclass{article}
\title{Informatik \rotatebox[origin=c]{180}{D}\raisebox{2pt}{:} -- Blatt 10}
\author{Rasmus Diederichsen}
\date{\today}
\usepackage[utf8]{inputenc}
\usepackage[ngerman]{babel}
\usepackage{microtype,
   booktabs,
   enumitem,
   amssymb,
   multirow,
   listings,
   cwpuzzle,
   tikz,
   multicol,
   IEEEtrantools,
   array,
   amsmath,
   amssymb,
   graphicx, 
lmodern}
\usepackage[pdftitle={Informatik D -- Blatt 10}, 
   pdfauthor={Rasmus Diederichsen}, 
   hyperfootnotes=true,
   colorlinks,
   bookmarksnumbered = true,
   linkcolor = blue,
   plainpages = false,
citecolor = blue]{hyperref}
\usepackage[T1]{fontenc}
\usepackage[all]{hypcap}
\renewcommand{\thesection}{}
\renewcommand{\thesubsection}{Aufgabe \arabic{section}.\arabic{subsection}}
\renewcommand{\thesubsubsection}{}
\setcounter{section}{9}

\begin{document}

\maketitle

\section{} 
\subsection{} 
\subsubsection{a)}

$\mathcal{A}$ berechnet die Funktion $f(x) = 3^{3^x}$, für $x > 0$.

\subsubsection{b)}

Das uniforme Kostenmaß ist nur sinnvoll, wenn primitive Operationen
(Addition, Multiplikation) wirklich in $\mathcal{O}(1)$ durchgeführt werden
können. Für reale Computer kann dies je nach Architektur nur für Zahlen $\le
2^{32}$ oder $2^{64}$ passieren, da die Register keine größeren Zahlen fassen
können.

Die Werte, die $y$ in $\mathcal{A}$ annimmt, laufen jedoch schon für relativ
kleine $x$ ($x \gtrapprox 3.366$ bei 64-Bit-Integralzahlen) über diesen Wertebereich und
sind dann nicht mehr in 32 oder 64 Bit darstellbar. Die arithmetischen
Operationen sind dann nicht mehr atomar und der Aufwand wächst mit der
Eingabegröße, ist also nicht mehr konstant.

\subsubsection{c)}

Da der Wert von $y$ in jedem Schleifendurchlauf mit 3 potenziert wird, braucht
Schleifendurchlauf $i$ $3(3^i\log{3})^2$ viele Schritte. Die Gesamtzahl an
Schritten ist somit
\begin{equation*}
   f(x) = \sum\limits_{i=1}^x 3(\log_2 3^{3^i})^2 = \sum\limits_{i=1}^x 3(3^i\log_2 3)^2\in
   \mathcal{O}\left(3^{2x}\right) = \mathcal{O}\left(9^x\right)
\end{equation*}

\subsection{} 

\subsubsection{a)}

Das zugehörige Entscheidungsproblem ist

\begin{center}
   \begin{tabular}{p{.6\textwidth}}
      Gegeben Konstanten $c, A, b$ sowie $k \in \mathbb{R}$, gibt es ein $x\in\mathbb{R}^n$, sodass $c^Tx \ge k$ mit $Ax \le b$?
   \end{tabular}
\end{center}

\subsubsection{b)}

\lstset{
   basicstyle=\footnotesize\ttfamily,
   language=Python,
   breaklines=true,
   commentstyle=\color{blue},
   keywordstyle=\color{purple}\textbf,
   numberstyle=\tiny\color{gray},
   numbers=left,
   stringstyle=\color{olive},
}
\begin{lstlisting}[escapeinside={*}{*},title={Optimierungsalgorithmus}]
def optimize(c, A, b):
   import numpy as np
   upperLim = np.sum(c) # groesser geht's nicht
   lowerLim = 0

   return binarySearch(0, upperLim)

def *$\mathcal{A}$*(k):
      ... # entscheide ob max >= k ist mit c, A, b

   def binarySearch(lower, upper):
      if (upper == lower) return upper
      if (*$\mathcal{A}$*(upper)):
         return upper # wenn es ein Element >= upper gibt, muss es == upper sein
      if (not *$\mathcal{A}$*(lower + 1)):
         return lower # wenn es kein Element > lower gibt, muss das maximum == lower sein
      if(*$\mathcal{A}$*(lower + (upper + lower) / 2)): # weiter in oberer Haelfte suchen
         return binarySearch(lower + (upper + lower) / 2, upper)
      else: # weiter in unterer Haelfte suchen
         return binarySearch(lower, upper - (upper + lower) / 2)
\end{lstlisting}

Im Algorithmus wird der Suchraum in jeder Iteration halbiert, es kann also
maximal $\log_2{n}$ viele rekursive Aufrufe geben, und auch das nur, falls $c =
\{1\}^n$. Für jeden Aufruf von \texttt{binarySearch} wird einmal $\mathcal{A}$
aufgerufen, die Gesamtlaufzeit ist somit
$\in\mathcal{O}\left(t_{\mathcal{A}}(n,m)\log_2{n}\right)$. Ist man auch an dem
zu dieser Lösung führenden $x$ interessiert, so errechnet man dies aus den $n$
Ungleichungen $Ax \le b$, mithilfe von $c^Tx = k$.

% the fuck happened to the space here?
\vspace{2\baselineskip}

\subsection{} 

\subsubsection{a)}

Man untersucht nacheinander alle Terme $T_i$ darauf, ob sie ein Literal und
seine Negation enthalten. Falls nein, so kann man alle Literale so belegen, dass
sie \texttt{true} sind und man terminiert, da nur ein $T_i$ \texttt{true} sein
muss. Falls ja, so geht man weiter zur nächsten Klausel.

Ist auch der letzte Term unerfüllbar
(der worst case), so hat man alle Literale nur einmal angeschaut, da es pro
Literal nur eine Möglichkeit gibt, dass es \texttt{true} wird. Der Aufwand ist
also maximal linear.

\subsubsection{b)}

\newcommand{\NP}{N\kern-2ptP}
Das Theorem gälte, wenn die Überführung einer beliebigen Formel in DNF auch in
polynomieller Zeit durchführbar wäre. Dieses Problem ist jedoch (anscheinend)
bewiesenermaßen $\NP$-schwer, und daher nur in $P$, wenn es auch in $\NP$ ist und $P =
\NP$ gilt. Dies ist nicht bewiesen, daher gilt der Beweis nicht.
\subsection{} 

\renewcommand{\labelenumi}{(\arabic{enumi})}
\begin{enumerate}
   \item $(x_{Stannis,1} \wedge x_{Sansa,1}) \vee (\neg x_{Stannis,1} \wedge
      \neg x_{Sansa,1})$
   \item $\bigwedge\limits_t\bigvee\limits_{v_1 \neq v_2} \neg(x_{v_1,t} \vee
      x_{v_2,t})$ 
      \begin{itemize}[label=$\rightarrow$]
         \item Für jeden Zeitpunkt müssen mindestens zwei ungleiche Personen
            \emph{nicht} in Tyrions Gemächern sein, über die anderen wird keine
            Aussage getroffen. 
      \end{itemize}
      Es resultieren folgende Klauseln:
      % a = []
      % 1.upto(3) do |t|
      %    arr = []
      %    names1 = ['Stannis', 'Sansa', 'Shae', 'Sandor']
      %    names2 = ['Stannis', 'Sansa', 'Shae', 'Sandor']
      %    names1.each do |v1|
      %       names2.each do |v2|
      %          if v1 != v2 
      %             arr.push "\\neg(x_\{#{v1},#{t}\} \\vee x_\{#{v2},#{t}\})"
      %          end
      %       end
      %       names2.delete v1
      %    end
      %    a.push arr
      % end

      % a.map! {|array| "(" + array.join(" \\vee ") + ")"}
      % puts "$", a.join(" \\wedge "), "$"

      $ (\neg(x_{Stannis,1} \vee x_{Sansa,1}) \vee \neg(x_{Stannis,1} \vee
      x_{Shae,1}) \vee \neg(x_{Stannis,1} \vee x_{Sandor,1}) \vee
      \neg(x_{Sansa,1} \vee x_{Shae,1}) \vee \neg(x_{Sansa,1} \vee x_{Sandor,1})
      \vee \neg(x_{Shae,1} \vee x_{Sandor,1})) \wedge (\neg(x_{Stannis,2} \vee
      x_{Sansa,2}) \vee \neg(x_{Stannis,2} \vee x_{Shae,2}) \vee
      \neg(x_{Stannis,2} \vee x_{Sandor,2}) \vee \neg(x_{Sansa,2} \vee
      x_{Shae,2}) \vee \neg(x_{Sansa,2} \vee x_{Sandor,2}) \vee \neg(x_{Shae,2}
      \vee x_{Sandor,2})) \wedge (\neg(x_{Stannis,3} \vee x_{Sansa,3}) \vee
      \neg(x_{Stannis,3} \vee x_{Shae,3}) \vee \neg(x_{Stannis,3} \vee
      x_{Sandor,3}) \vee \neg(x_{Sansa,3} \vee x_{Shae,3}) \vee \neg(x_{Sansa,3}
      \vee x_{Sandor,3}) \vee \neg(x_{Shae,3} \vee x_{Sandor,3})) $

   \item $\bigwedge\limits_{t_1\neq t_2} \neg(x_{Stannis,t_1} \wedge
      x_{Stannis,t_2}) \wedge \neg(x_{Shae,t_1} \wedge x_{Shae,t_2})$

      Es resultieren die Klauseln:

      % a = []
      % times1 = (1..3).to_a
      % times2 = (1..3).to_a
      % times1.each do |t1|
      %    times2.each do |t2|
      %       a.push "\\neg(x_\{Stannis,#{t1}\} \\wedge x_\{Stannis,#{t2}\}) \\wedge \\neg(x_\{Shae,#{t1}\} \\wedge x_\{Shae,#{t2}\})" if t1 != t2
      %    end
      %    times2.delete t1
      % end
      % puts "$" + a.join("\\wedge") + "$"
      $\neg(x_{Stannis,1} \wedge x_{Stannis,2}) \wedge \neg(x_{Shae,1} \wedge
      x_{Shae,2})\wedge\neg(x_{Stannis,1} \wedge x_{Stannis,3}) \wedge
      \neg(x_{Shae,1} \wedge x_{Shae,3})\wedge\neg(x_{Stannis,2} \wedge
      x_{Stannis,3}) \wedge \neg(x_{Shae,2} \wedge x_{Shae,3})$
   \item $(x_{Sansa,1} \wedge x_{Sansa,2}) \vee \neg(x_{Sansa,1} \vee
      x_{Sansa,2}) \wedge \neg x_{Sansa,3}$
   \item $(x_{Sansa,2} \wedge x_{Shae,2}) \vee \neg(x_{Sansa,2} \vee x_{Shae,2})$
      \begin{itemize}[label=$\rightarrow$]
         \item Sansa und Shae müssen sich entweder beide in Tyrions Gemächern
            aufgehalten haben, oder sie waren beide nicht dort. Zwar müssten sie
            dann \emph{beide} am selben anderen Ort gewesen sein, dies spielt
            hier aber keine Rolle und ist nicht direkt darstellbar, da die
            einzigen Orte Tyrions Gemächer und Nicht-Tyrions-Gemächer sind.
      \end{itemize}
   \item $x_{Shae,2}$
   \item $\bigwedge\limits_t x_{Shae,t} \rightarrow y_t$. Dies lässt sich
      umformen zu $\bigwedge\limits_t \neg x_{Shae,t} \vee y_t$.
      Es resultieren die Formeln

      $(\neg x_{Shae,1} \vee y_1) \wedge (\neg x_{Shae,2} \vee y_2) \wedge (\neg
      x_{Shae,3} \vee y_3)$
   \item[(8*)] $\bigwedge\limits_{t_2 > t_1} \neg y_{t_1} \rightarrow \neg y_{t_2}$,
      was äquivalent ist zu $\bigwedge\limits_{t_2 > t_1} y_{t_1} \vee \neg
      y_{t_2}$. 
      Es resultieren die Klauseln

      $(y_{1} \vee \neg y_{2}) \wedge (y_{1} \vee \neg y_{3}) \wedge (y_{2} \vee
      \neg y_{3})$
\end{enumerate}

Wir sind nun in der Position, einige Schlussfolgerungen zu machen.
\renewcommand{\labelenumi}{\arabic{enumi}.}
\begin{enumerate}
   \item Wegen (6) wissen wir, dass $x_{Shae,2}=\texttt{true}$.
   \item Wegen (7) und 1. wissen wir, dass $y_2 = \texttt{true}$.
   \item Wegen (5) und 1. wissen wir, dass nicht nur $x_{Shae,2}=\texttt{true}$,
      sondern auch $x_{Sansa,2}=\texttt{true}$, da beide am gleichen Ort waren.
   \item Wegen (4) und 3. wissen wir, dass $x_{Sansa,1}=\texttt{true}$.
   \item Wegen (1) und 4. wissen wir, dass $x_{Stannis,1}=\texttt{true}$, da
      beide am gleichen Ort waren.
   \item Wegen (3) und 5. wissen wir, dass $x_{Stannis,2}=\texttt{false}$. 
   \item Wegen (2) und 3. wissen wir auch, dass $x_{Sandor,2}=\texttt{false}$ 
      Alle Konjunkte
      müssen wahr sein, im zweiten Konjunkt sind jedoch wegen 3. alle Disjunkte
      \texttt{false}, die Shae und Sansa enthalten. Also muss
      $\neg(x_{Stannis,2} \vee x_{Sandor,2})$ \texttt{true} sein, mithin
      $x_{Stannis,2}=\texttt{false}$ und $x_{Sandor,2}=\texttt{false}$. 
   \item Wegen (3) und 6. wissen wir, dass $x_{Stannis,3}=\texttt{false}$.
   \item Wegen (3) und 1. wissen wir, dass $x_{Shae,3}=\texttt{false}$.
   \item Wegen (4) wissen wir, dass $x_{Sansa,3}=\texttt{false}$
   \item Der einzige, der zur Stunde 3 bei Tyrion gewesen sein kann, war Sandor.
\end{enumerate}

Der Mörder muss also Sandor Clegane sein.

\subsection{} 

\subsubsection{a)}

Es bezeichne $H$ einen Haufen mit Hinkelsteinen, die die Farben $a,\ldots,c$
annehmen können. Wir verwenden die Variable $x_f$, die \texttt{true} ist, genau
dann wenn $f \in F_{schön}$.

Weil eine Unterschiedliche Anzahl Steine auf einem Haufen
liegen können, ergeben sich folgende Möglichkeiten :

\begin{equation*}
   \left.
   \begin{minipage}{.7\textwidth}
      \begin{itemize}[align=left]
         \item[$H=\{a,b,c\}$] $(\neg x_a \vee x_b \vee x_c) \wedge (x_a \vee \neg x_b
            \vee x_c) \wedge (x_a \vee x_b \vee \neg x_c)$. Die Möglichkeit, dass
            alle 3 Farben schön sind, ist hierbei ebenfalls abgedeckt.
         \item[$H=\{a,a,b\}$:] $(x_b \rightarrow x_a)$
         \item[$H=\{a,a,a\}$:] $(x_a \vee \neg x_a) \equiv \texttt{true}$
         \item[$H=\{a,b\}$:] $(x_a \wedge x_b) \vee (\neg x_a \wedge \neg x_b)$
         \item[$H=\{a,a\}$:] $(x_a \vee \neg x_a) \equiv \texttt{true}$
         \item[$H=\{a\}$:] $\neg x_a$ (da bei nur einem Stein dessen Farbe nicht schön
            sein darf)
         \item[$H=\emptyset$:] trivialerweise \texttt{true}, da dieser Haufen nichts
            zur Lösung beiträgt.
      \end{itemize}
   \end{minipage}\quad
\right\}\bigwedge \wedge x_{\ell}
\end{equation*}

\subsubsection{b)}

Ein Algorithmus zur Lösung kann so aussehen:

\begin{lstlisting}[escapeinside={*}{*}]
F*$_{schön}$* = {'a','b',...}
while(*$\exists$* Haufen mit genau einem schoenen Stein mit Farbe x):
   F*$_{schön}$* = F*$_{schön} \setminus $*{x}
   loesche diesen Stein von Haufen
if(*$\ell\in$* F*$_{schön}$*):
   return true
else 
   return false
\end{lstlisting}

\subsection{} 

Für $\chi = \text{SAT}$ ist die Lösung relativ simpel. Sei $n$ die Anzahl der
vorkommenden Variablen. Die folgende Funktion gibt nach Eingabe einer
Klauselmenge mithilfe von $\mathcal{A}_\chi$ einen Zeugen für $\mathcal{I}^*$ in
Form einer Assoziationsliste zurück.

Hierbei sei \texttt{solvedp} ein Prädikat, dass bestimmt, ob ein Satz Formeln
mit gegebenen Variablenzuweisungen bereits erfüllt ist, und \texttt{variables}
retourniere die Liste von Variablen der SAT-Instanz.

\begin{lstlisting}[language=Lisp,escapeinside={|}{|}]
(defun witness (problem)
   (labels ((solve (formulas cur_assignments)
      (cond ((solvedp formulas cur_assignments) cur_assignments)
            ((not (|$\mathcal{A}_{\chi}$| formulas cur_assignments)) nil)
            (t (or (solve formulas 
                          (cons (cons (find-if #'unassignedp 
                                               (variables formulas))
                                      t) 
                                cur_assignments))
                   (solve formulas 
                          (cons (cons (find-if #'unassignedp 
                                               (variables formulas)) 
                                      nil) 
                                cur_assignments)))))))
      (solve problem nil)))
\end{lstlisting}

In Worten geht man wie folgt vor. Man betrachtet nacheinander alle in
$\mathcal{I}^*$ vorkommenden Variablen $x_1,\ldots,x_n$. Man setzt nun die
aktuelle Variable $x_i$ auf \texttt{true}, reduziert die Klauselmenge, sodass sie
wieder nur Variablen enthält und prüft, ob das um eine Variable reduzierte
Problem noch immer lösbar ist. 

Falls ja, so behält man die soeben getroffene
Zuweisung und wendet sich der nächsten freien Variable zu. 

Falls nein, so wählt man stattdessen $x_i = \texttt{false}$, reduziert die
Klauselmenge und geht zur nächsten Variable (das reduzierte Problem ist nun auf
jeden Fall lösbar, da wir bereits wissen, dass die vorliegende eine
Ja-Instanz ist).

Irgendwann ist man beim trivialen Problem \texttt{true} angekommen und weiß nun,
dass die gespeicherten Zuweisungen einen Zeugen von $\mathcal{I}^*$ darstellen.
Ob es noch andere gibt, weiß man nicht.

Dieses Schema ließe sich theoretisch auch auf andere Probleme anwenden. Diese
könnte man entweder auf SAT reduzieren (weil SAT $\NP$-vollständig ist), oder
die Reduktion des Problems geschieht auf andere Weise. Im Graph Coloring-Problem
würde man einem beliebigen Knoten eine Farbe zuordnen und ihn aus dem Graphen
entfernen und für alle Nachbarknoten seine Farbe aus der Menge erlaubter Farben
entfernen. Wiederum prüft man mit $\mathcal{A}_\chi$, ob der geschrumpfte Graph
noch colorierbar ist. Falls ja, geht man vor wie mit dem ersten Knoten, falls
nein, so backtracked man und wählt die nächste Farbe für den ersten Knoten.
\end{document}
