\documentclass{article}
\title{Informatik \rotatebox[origin=c]{180}{D}\raisebox{2pt}{:} -- Blatt 7}
\author{Rasmus Diederichsen}
\date{\today}
\usepackage[utf8]{inputenc}
\usepackage[ngerman]{babel}
\usepackage[table,svgnames]{xcolor}
\usepackage{wasysym}
\let\iiint\relax % wasysym-amsmath interference
\let\iint\relax
\usepackage{microtype,
   booktabs,
   enumitem,
   multirow,
   listings,
   cwpuzzle,
   tikz,
   multicol,
   IEEEtrantools,
   array,
   amsmath,
   amssymb,
   graphicx, 
lmodern}
\usepackage[pdftitle={Informatik D -- Blatt 7}, 
   pdfauthor={Rasmus Diederichsen}, 
   hyperfootnotes=true,
   colorlinks,
   bookmarksnumbered = true,
   linkcolor = blue,
   plainpages = false,
citecolor = blue]{hyperref}
\usepackage[T1]{fontenc}
\usepackage[all]{hypcap}
\renewcommand{\thesection}{}
\renewcommand{\thesubsection}{Aufgabe \arabic{section}.\arabic{subsection}}
\renewcommand{\thesubsubsection}{}
\setcounter{section}{6}

\begin{document}

\maketitle

\section{} 
\subsection{} 
\subsubsection{a)}

Es ist $L = \{a^kb^lc^m \mid k,l,m\in\mathbb{N}\} \cup \{d^ja^kb^kc^k \mid
j,k\in\mathbb{N}, j > 0\} = L_1 \cup L_2$. Offensichtlich ist $L_2$ nicht
kontextfrei, da es sich in einer kontextfreien Grammatik nicht sicherstellen lässt,
dass mehr als zwei verschieden Symbole gleich häufig vorkommen müssen. Dies wäre
unproblematisch, falls $L_2 \subseteq L_1$, denn eine reguläre Sprache (wie
$L_1$) kann ja ohne Weiteres eine Sprache höherer Klasse als Teilmenge
beinhalten. Dies ist hier nicht der Fall, $L_1 \cap L_2 = \emptyset$. Um $L$
darzustellen, ist also eine kontextfreie Grammatik nicht ausreichend.

\subsubsection{b)}

Es wird hier angenommen, dass es sich in der Sprachspezifikation um ein
nicht-exklusives 'Oder' handelt.

\let\oldlabelenumi\labelenumi
\renewcommand{\labelenumi}{\textbf{Fall \arabic*:}}
\begin{enumerate}[leftmargin=*]
   \item Für $j = 0, k=l=m$ wähle $v,x$ entweder so, dass sie jeweils nur ein
      Zeichen (beliebig oft wiederholt) enthalten oder leer sind (nicht beide
      zugleich). Man kann dann beliebig aufpumpen. Zwar verliert man die
      Struktur $a^kb^kc^k$, jedoch ist $a^lb^mc^n$ ja immernoch $\in L$.
   \item 
      Falls $j=0,k,l,m \ge 0$, ist die Sprache offensichtlich kontextfrei (sie
      ist sogar regulär), da es keine weiteren Beschränkungen für $k,l,m$ gibt. Das
      Pumping Lemma hilft hier also schonmal nicht weiter.
   \item Falls $j \neq 0, k, l, m \ge 0$, ist die Sprache ebenfalls regulär und
      damit kontextfrei.
   \item
      Falls $j \neq 0, k = l = m$ haben wir ein Wort $z =
      d^ja^kb^kc^k$ mit $j+3k\ge n$. In diesem Fall wähle
      \begin{itemize}
         \item $u=\varepsilon$
         \item $v = d$
         \item $w=\varepsilon$
         \item $x=\varepsilon$
         \item $y=a^kb^kc^k$
      \end{itemize}
      Die Bedingungen $|vx|\ge 1$ und $|vwx| \le n$ sind erfüllt.
      Offensichtlich ist $uv^iwx^iy = \varepsilon d^i\varepsilon
      \varepsilon^ia^kb^kc^k \in L$. Diese Zerlegung lässt sich immer finden,
      unabhäbgig von $n$. Das Pumping Lemma ist demnach hier unnütz.

\end{enumerate}

\subsection{} 

Die Sprache $L$ ist nicht kontextfrei, was mir mittels des Pumping Lemmas
zeigen. Sei $n$ die Wortmindestlänge und $z_n = a^nb^{2^n} \in L$. Es gilt
\begin{equation*}
   |z_n| = n + 2^n \text{ und } |z_{n+1}| = n + 1 + 2^{n+1}
\end{equation*}

Es gibt folgende Möglichkeiten, eine Zerlegung $uvwxy$ von $z_n$ zu finden:
\begin{enumerate}[leftmargin=*]
   \item $vwx = a^k, k \ge 1$ enthält nur $a$s. Hierbei könten durch Pumpen beliebig viele $a$s
      erzeugt werden, ohne dass die $b$s mitgepumpt werden daher z.B. $uv^2wx^2y\not\in L$.
   \item $vwx$ enthält $a$s und $b$s.
      \begin{enumerate}
         \item Falls $w = \varepsilon$, so ist $vwx = a^lb^k,n \ge l+k \ge 1$.
            Es gilt für $uv^2wx^2y = a^{n-l}a^{2l}b^{2k}b^{2^n-k} =
            a^{n+l}b^{2^n+k}$ dass
            $|uv^2wx^2y| = n + l + 2^n + k$. Dieses Wort muss mindestens die
            Länge von $z_{n+1}$ haben, um in $L$ enthalten sein zu können. Die
            Gleichung
            \begin{IEEEeqnarray*}{rCl}
               n + l + 2^n + k & = & n + 1 + 2^{n+1} \\
               l + k & = & 1 + 2^{n+1} -2^n \\
                     & = & 1 + 2^n(2 - 1)\\
                     & = & 1 + 2^n
            \end{IEEEeqnarray*}
            ist jedoch nicht erfüllbar, da $l + k \le n$ nach Vorraussetzung.
         \item Falls $w \neq \varepsilon$, ändert dies nichts, da $l + k$
            dadurch lediglich kleiner würde und die obige Gleichung weiterhin
            unerfüllbar ist.
      \end{enumerate}
   \item $vwx$ enthält nur $b$s. Mit der Begründung von Punkt 1 gilt $uv^iwx^iy
      \not\in L$
\end{enumerate}

\subsection{} 

\newcommand{\grule}[2]{#1 & \rightarrow & #2}
\begin{multicols}{2}

   Ausgehend von 
   \begin{equation*}
      \begin{array}{rcl}
         \grule{S}{E \mid ABC} \\
         \grule{A}{bdC \mid E \mid S} \\
         \grule{B}{acDC \mid C} \\
         \grule{C}{ab \mid Ja \mid IdA} \\
         \grule{D}{deF \mid DeF \mid dEF \mid DEF} \\
         \grule{E}{abba \mid aBBa \mid AbbA} \\
         \grule{F}{JcJ \mid GJ \mid bG} \\
         \grule{G}{F \mid IG} \\
         \grule{H}{SA \mid Hcc} \\
         \grule{I}{c \mid cIc \mid dF} \\
      \end{array}
   \end{equation*}


   Markieren wir

   \newcommand{\marked}[1]{\colorbox{yellow}{\hspace{-2pt}$#1$\hspace{-2pt}}}

   \subsubsection{Schritt 1}
   \begin{equation*}
      \begin{array}{rcl}
         \grule{S}{E \mid ABC} \\
         \grule{A}{bdC \mid E \mid S} \\
         \grule{B}{acDC \mid C} \\
         \grule{\marked{C}}{\marked{ab} \mid Ja \mid IdA} \\
         \grule{D}{deF \mid DeF \mid dEF \mid DEF} \\
         \grule{E}{abba \mid aBBa \mid AbbA} \\
         \grule{F}{JcJ \mid GJ \mid bG} \\
         \grule{G}{F \mid IG} \\
         \grule{H}{SA \mid Hcc} \\
         \grule{I}{c \mid cIc \mid dF} \\
      \end{array}
   \end{equation*}

   \vfill{ }

   \subsubsection{Schritt 2}
   \begin{equation*}
      \begin{array}{rcl}
         \grule{S}{E \mid AB\marked{C}} \\
         \grule{A}{bd\marked{C} \mid E \mid S} \\
         \grule{\marked{B}}{acD\marked{C} \mid \marked{C}} \\
         \grule{\marked{C}}{\marked{ab} \mid Ja \mid IdA} \\
         \grule{D}{deF \mid DeF \mid dEF \mid DEF} \\
         \grule{E}{abba \mid aBBa \mid AbbA} \\
         \grule{F}{JcJ \mid GJ \mid bG} \\
         \grule{G}{F \mid IG} \\
         \grule{H}{SA \mid Hcc} \\
         \grule{I}{c \mid cIc \mid dF} \\
      \end{array}
   \end{equation*}

   \subsubsection{Schritt 3}

   \begin{equation*}
      \begin{array}{rcl}
         \grule{S}{E \mid A\marked{B}\marked{C}} \\
         \grule{A}{bd\marked{C} \mid E \mid S} \\
         \grule{\marked{B}}{acD\marked{C} \mid \marked{C}} \\
         \grule{\marked{C}}{\marked{ab} \mid Ja \mid IdA} \\
         \grule{D}{deF \mid DeF \mid dEF \mid DEF} \\
         \grule{\marked{E}}{\marked{abba}\mid \marked{aBBa} \mid AbbA} \\
         \grule{F}{JcJ \mid GJ \mid bG} \\
         \grule{G}{F \mid IG} \\
         \grule{H}{SA \mid Hcc} \\
         \grule{I}{c \mid cIc \mid dF} \\
      \end{array}
   \end{equation*}

   \subsubsection{Schritt 4}

   \begin{equation*}
      \begin{array}{rcl}
         \grule{\marked{S}}{\marked{E} \mid A\marked{B}\marked{C}} \\
         \grule{\marked{A}}{bd\marked{C} \mid \marked{E} \mid S} \\
         \grule{\marked{B}}{acD\marked{C} \mid \marked{C}} \\
         \grule{\marked{C}}{\marked{ab} \mid Ja \mid IdA} \\
         \grule{D}{deF \mid DeF \mid d\marked{E}F \mid D\marked{E}F} \\
         \grule{\marked{E}}{\marked{abba}\mid \marked{aBBa} \mid AbbA} \\
         \grule{F}{JcJ \mid GJ \mid bG} \\
         \grule{G}{F \mid IG} \\
         \grule{H}{SA \mid Hcc} \\
         \grule{I}{c \mid cIc \mid dF} \\
      \end{array}
   \end{equation*}

   \subsubsection{}

   Wir sehen also, dass die Sprache nicht leer ist. Markiert man bis zum Ende
   durch, erhält man
   \begin{equation*}
      \begin{array}{rcl}
         \grule{\marked{S}}{\marked{E} \mid \marked{ABC}} \\
         \grule{\marked{A}}{bd\marked{C} \mid \marked{E} \mid \marked{S}} \\
         \grule{\marked{B}}{acD\marked{C} \mid \marked{C}} \\
         \grule{\marked{C}}{\marked{ab} \mid Ja \mid \marked{IdA}} \\
         \grule{D}{deF \mid DeF \mid d\marked{E}F \mid D\marked{E}F} \\
         \grule{\marked{E}}{\marked{abba}\mid \marked{aBBa} \mid \marked{AbbA}} \\
         \grule{F}{JcJ \mid GJ \mid bG} \\
         \grule{G}{F \mid \marked{I}G} \\
         \grule{\marked{H}}{\marked{SA} \mid \marked{Hcc}} \\
         \grule{\marked{I}}{\marked{c} \mid \marked{cIc} \mid dF}
      \end{array}
   \end{equation*}
\end{multicols}

Die reduzierte Grammatik ist also 

\begin{equation*}
   \begin{array}{rcl}
      \grule{S}{E \mid ABC} \\
      \grule{A}{E \mid S} \\
      \grule{B}{C} \\
      \grule{C}{ab \mid IdA} \\
      \grule{E}{abba \mid aBBa \mid AbbA} \\
      \grule{I}{c \mid cIc}
   \end{array}
\end{equation*}
Die Regeln der Form $V \rightarrow V$ sind $\{S\rightarrow E, A \rightarrow E, A
\rightarrow S, B \rightarrow C\}$. Hieraus entsteht der Graph

\usetikzlibrary{arrows,automata,positioning}
\tikzset{every node/.append style={font=\sf}}

\begin{center}
   \begin{tikzpicture}[auto,thick,>=stealth',
      initial text={\sffamily start},node distance=1cm]
      \node[state] (S) {$S$};
      \node[state] (A) [above right=of S] {$A$};
      \node[state] (E) [below right=of S] {$E$};
      \node[state] (B) [right=of A] {$B$};
      \node[state] (C) [right=of E] {$C$};
      \path[->]
      (A) edge (S)
      (S) edge (E)
      (A) edge (E)
      (B) edge (C);
   \end{tikzpicture}
\end{center}

Der Graph ist bereits kreisfrei. Es müssen also noch die Senken eliminiert
werden.

\subsubsection{Elimination von Senke $C$}

Man führt die neuen Regeln
\begin{equation*}
   \begin{array}{rcl}
      \grule{B}{ab \mid IdA}
   \end{array}
\end{equation*}
ein. Entfernt wird die Regel $B \rightarrow C$.

\subsubsection{Elimination von Senke $E$}

Man führt die neuen Regeln
\begin{equation*}
   \begin{array}{rcl}
      \grule{S}{abba \mid aBBa \mid AbbA} \\
      \grule{A}{abba \mid aBBa \mid AbbA}
   \end{array}
\end{equation*}
ein. Entfernt werden die Regeln $A \rightarrow E$ und $S \rightarrow E$. Der reduzierte Graph ist

\begin{center}
   \begin{tikzpicture}[auto,thick,>=stealth',
      initial text={\sffamily start},node distance=1cm]
      \node[state] (A) {$A$};
      \node[state] (S) [right=of A] {$S$};
      \path[->]
      (A) edge (S);
   \end{tikzpicture}
\end{center}

\subsubsection{Elimination von Senke $S$}

Man führt die Regeln
\begin{equation*}
   \begin{array}{rcl}
      \grule{A}{abba \mid aBBa \mid AbbA \mid ABC}
   \end{array}
\end{equation*}
ein, von denen Teile in $A$ bereits enthalten sind.
Wir entfernen $A \rightarrow S$. Die finale Grammatik ist
\begin{equation*}
   \begin{array}{rcl}
      \grule{S}{abba \mid aBBa \mid AbbA \mid ABC} \\
      \grule{A}{abba \mid aBBa \mid AbbA \mid ABC} \\
      \grule{B}{ab \mid IdA} \\
      \grule{C}{ab \mid IdA} \\
      \rowcolor{lightgray!50} \grule{E}{abba \mid aBBa \mid AbbA}\\
      \grule{I}{c \mid cIc}
   \end{array}
\end{equation*}
wobei die Variable $E$ nicht mehr erreicht wird aufgrund der Subsitutionen mit
ihrer rechten Seite in den anderen Regeln.

Der Graph bestehend aus den Variablen ist
\begin{center}
   \begin{tikzpicture}[auto,thick,>=stealth',
      initial text={\sffamily start},node distance=2cm]
      \node[state,initial] (S) {$S$};
      \node[state] (A) [right=of S]{$A$};
      \node[state] (B) [below left=of A] {$B$};
      \node[state] (C) [right=of A] {$C$};
      \node[state] (I) [right=of B] {$I$};
      \path[->]
      (S) edge (A)
      (S) edge (B)
      (S) edge[bend left=40,looseness=1.1,out=60,in=120] (C)
      (C) edge (I)
      (C) edge[bend left,color=red] (A)
      (A) edge[bend left,color=red] (C)
      (A) edge[bend left,color=red] (B)
      (B) edge[bend left,color=red] (A)
      (B) edge (I)
      (A) edge[loop above,color=red] ()
      (I) edge[loop above,color=red] ();
   \end{tikzpicture}
\end{center}
Dieser Graph hat Kreise, also ist die Sprache nicht endlich.

\subsection{} 

Das ``Problem'' mit der zweiten Definition ist, dass $A$ nur eine Variable sein
kann, der Rest der Satzform muss auf $u$ und $w$ aufgeteilt werden, die auf der
rechten Seite unverändert bestehen bleiben müssen. Egal ob man $A$ mit $A$ oder
$B$ belegt, die jeweils andere Variable wird nach anwenden der Produktion
rechts oder links stehen bleiben.

Um diese Art von Regeln darstellen zu können, führen wir ein:
\begin{equation*}
   \begin{array}{rcl}
      \grule{X}{AB} \\
      \grule{Y}{BA}
   \end{array}
\end{equation*}
ein, die beide der zweiten Definition Genüge tun. Zusätzlich definiert man die
Regel
\begin{equation*}
   X\rightarrow Y
\end{equation*}
sowie für jede Regel
\begin{equation*}
   uVv \rightarrow u\ldots AB \ldots v
\end{equation*}
eine Regel
\begin{equation*}
   uVv \rightarrow u\ldots X \ldots v
\end{equation*}

Alternativ könnte man Regeln der folgenden Form einführen ($C$ sei eine neue
Variable):

\begin{equation*}
   \begin{array}{rcl}
      \grule{AB}{AC} \\
      \grule{AC}{BC} \\
      \grule{BC}{BA}
   \end{array}
\end{equation*}

\subsection{} 

\vspace{\baselineskip}
\hspace{1.8cm}\begin{minipage}{.4\textwidth}
   \def\PuzzleSolutionContent#1{\makebox(1,1){\itshape{#1}}}
   \renewcommand{\PuzzleLineThickness}{1pt}
   \PuzzleSolution
   \begin{Puzzle}{5}{9}
      |*|*|b|c|a|c|c|a|*|.
      |*|*|*|c|*|*|*|c|*|.
      |*|c|b|c|a|a|c|b|c|.
      |*|*|*|c|*|a|c|a|b|.
      |b|c|b|c|b|c|b|c|a|.
   \end{Puzzle}
\end{minipage}
\vspace{-2cm}

\subsection{} 

\begin{center}
   \includegraphics{niemand_kehrt.jpeg}
\end{center}
\end{document}
