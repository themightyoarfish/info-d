\documentclass{article}
\title{Informatik \rotatebox[origin=c]{180}{D}\raisebox{2pt}{:} -- Blatt 8}
\author{Rasmus Diederichsen}
\date{\today}
\usepackage[utf8]{inputenc}
\usepackage[ngerman]{babel}
\usepackage[table,svgnames]{xcolor}
\usepackage{wasysym}
\let\iiint\relax % wasysym-amsmath interference
\let\iint\relax
\usepackage{microtype,
   booktabs,
   enumitem,
   multirow,
   listings,
   cwpuzzle,
   tikz,
   multicol,
   IEEEtrantools,
   array,
   amsmath,
   amssymb,
   graphicx, 
lmodern}
\usepackage[pdftitle={Informatik D -- Blatt 8}, 
   pdfauthor={Rasmus Diederichsen}, 
   hyperfootnotes=true,
   colorlinks,
   bookmarksnumbered = true,
   linkcolor = blue,
   plainpages = false,
citecolor = blue]{hyperref}
\usepackage[T1]{fontenc}
\usepackage[all]{hypcap}
\renewcommand{\thesection}{}
\renewcommand{\thesubsection}{Aufgabe \arabic{section}.\arabic{subsection}}
\renewcommand{\thesubsubsection}{}
\setcounter{section}{7}

\begin{document}

\maketitle

\section{} 
\subsection{} 

Den abgerundeten Logarithmus zu berechnen, ist dasselbe, wie den Index
(nullbasiert) des höchsten gesetzten Bits zu bestimmen. Da wir annehmen, dass
$\alpha$ keine führenden Nullen hat, ist dies gleichzeitig das Bit am linken
Ende der Eingabe. Die Anzahl an zu schreibenden $a$s ist also gleich dem Index
des höchstwertigsten Bits. Die Turingmaschine hierzu ist (z.B.) folgende:

\usetikzlibrary{automata,positioning,arrows}
\tikzset{every node/.append style={font=\sf}}
\tikzset{auto,thick,>=stealth',initial text={\sffamily start},node distance=2cm}

\vspace{\baselineskip}

\colorbox{teal!10}{Sei $\Sigma=\{0,1\}$}
\begin{center}
   \begin{tikzpicture}
      \node[state,initial] (S) {$S$};
      \node[state] (A) [right=of S] {$A$};
      \node[state] (B) [right=of A] {$B$};
      \node[state,accepting] (C) [right=of B] {$C$};
      \path[->]
      (S) edge[loop above] node {$\star\in\Sigma/\star,R$} ()
      (S) edge node {$\Box/\Box,L$} (A)
      (A) edge node {$\star\in\Sigma/\Box,L$} (B)
      (B) edge[loop above] node {$\star\in\Sigma/a,L$} ()
      (B) edge node {$\Box/\Box,R$} (C);
   \end{tikzpicture}
\end{center}

Hierbei ist die Kante zwischen $A$ und $B$ dafür zuständig, eine Stelle der
Eingabe abzuschneiden, da man ein $a$ weniger braucht, als die Eingabe lang ist.

\subsection{} 
\subsubsection{a)}

\lstset{
   basicstyle=\footnotesize\ttfamily,
   language=C,
   breaklines=true,
   commentstyle=\color{blue},
   keywordstyle=\color{purple}\textbf,
   numberstyle=\tiny\color{gray},
   numbers=left,
   stringstyle=\color{olive},
}
\lstset{
  literate={ö}{{\"o}}1
           {ä}{{\"a}}1
           {ü}{{\"u}}1
           {ß}{{\ss}}1
}

Man geht wie folgt vor
\begin{center}
\begin{lstlisting}
   Laufe nach ganz R
   Ersetze Blank durch $, gehe R
   Ersetze Blank durch a, gehe L // +1
   Gehe nach ganz L // stehe auf Blank
   Gehe R
   Falls $ gelesen: // passiert nicht in 1. Iteration
      ersetze durch Blank, gehe R
      terminiere
   sonst: // a gelesen
      Ersetze a durch Blank // löschen
      Gehe R, bis $ gelesen
      Gehe R, solange a gelesen
      Ersetze Blank durch a, gehe R // 3 as schreiben
      Ersetze Blank durch a, gehe R 
      Ersetze Blank durch a, gehe L
      Goto Stelle 4
\end{lstlisting}
\end{center}

Die Turingmaschine hierzu lautet:

\tikzset{node distance=1.2cm}
\vspace{\baselineskip}
\colorbox{teal!10}{Sei $\Gamma=\{a\}$}
\begin{center}
   \begin{tikzpicture}
      \node[state,initial] (A) {$A$};
      \node[state] (B) [above=of A] {$B$};
      \node[state] (C) [right=of B] {$C$};
      \node[state] (D) [right=of C] {$D$};
      \node[state] (E) [right=of D] {$E$};
      \node[state] (F) [below=of E] {$F$};
      \node[state] (G) [below=of F] {$G$};
      \node[state] (H) [below=of G] {$H$};
      \node[state] (I) [left=of H] {$I$};
      \node[state] (J) [above=of I] {$J$};
      \node[state,accepting] (K) [above=of J] {$K$};

      \path[->]
      (A) edge[loop below] node {$a/a,R$} ()
      (A) edge node {$\Box/\$,R$} (B)
      (B) edge node {$\Box/a,L$} (C)
      (C) edge[loop above] node {$\star\in\Gamma\cup\{\$\}/\star,L$} ()
      (C) edge node {$\Box/\Box,R$} (D)
      (D) edge node {$a/\Box,R$} (E)
      (E) edge[loop above] node {$a/a,R$} ()
      (E) edge[loop right] node {$\$/\$,R$} ()
      (E) edge node {$\Box/a,R$} (F)
      (F) edge node {$\Box/a,R$} (G)
      (G) edge node {$\Box/a,L$} (H)
      (H) edge[loop right] node {$a/a,L$} ()
      (H) edge node {$\$/\$,L$} (I)
      (I) edge[out=180,in=-90,looseness=1] node {$a/a,L$} (C)
      (I) edge node {$\Box/\Box,R$} (J)
      (J) edge node {$\$/\Box,R$} (K);
   \end{tikzpicture}
\end{center}

\subsubsection{b)}

Zunächst wird mittels folgendem Automaten bestimmt, ob die Eingabe gerade oder
ungerade Länge hat:

\begin{center}
   \begin{tikzpicture}[node distance=1.5cm]
      \node[state,initial] (A) {$A$};
      \node[state] (B) [below=of A] {$B$};
      \node[state] (C) [right=of A] {$C$};
      \node[state] (D) [above=of C] {$D$};
      \node[state,dashed] (E) [right=of B] {$E$};
      \node[state,dashed] (F) [above=of A] {$F$};

      \path[->]
      (A) edge node[left] {$\Box/\Box,L$} (B)
      (B) edge[loop left] node {$a/a,L$} ()
      (B) edge node {$\Box/\Box,R$} (E)
      (A) edge[bend left] node {$a/a,R$} (C)
      (C) edge[bend left] node {$a/a,R$} (A)
      (C) edge node[right] {$\Box/\Box,L$} (D)
      (D) edge[loop above] node {$a/a,L$} ()
      (D) edge node {$\Box/\Box,R$} (F);
   \end{tikzpicture}
\end{center}

Hierbei folgt an Knoten $F$ der Automat, der ungerade Eingaben verarbeitet, und
an $E$ der für gerade Eingaben. Im Weiteren geht man dann wie folgt vor:

\begin{lstlisting}
   Bestimme, ob Eingabe gerade oder ungerade
   Falls ungerade
      gehe vor gemäß a)
   sonst
      Laufe nach ganz R // stehe auf Blank
      Ersetze Blank durch $ // markiere Output-Buffer
      Laufe nach ganz L // stehe auf Blank
      Gehe R
      Falls $ gelesen, ersetze durch Blank, gehe R
         terminiere
      sonst // stehe auf a
         Ersetze a durch Blank, gehe R
         Ersetze a durch Blank, gehe R // min 2 as, da gerade und > 0
         Laufe ganz R bis $ gelesen // bis zum Output-Buffer
         Laufe R solange a gelesen // zum Ende des Buffers
         Ersetze Blank durch a, gehe L // zwei a gelöscht, eins geschrieben
         Goto Stelle 7
\end{lstlisting}

\subsection{} 

Eine Turingmaschine $\mathcal{M}$, die nur $n$ Felder des Bandes beschreibt, kann eine solche
simulieren, die $c\cdot n$ Felder beschreibt und das Alphabet $\Gamma$ besitzt,
wenn sie ein Alphabet $\Gamma_\mathcal{M} = \{k \mid k \in \Gamma^c\}$ besitzt,
ihr Alphabet also eine Menge von $c$-Tupeln aus $\Gamma$ ist. Dies ist effektiv
das selbe wie eine mehrbändige Turingmaschine mit jeweile endlichem Speicher, da
in einem Übergang einzelne Einträge des Tupels geändert werden können. Der
Automat wird dafür mehr Zustände benötigen.

\subsection{} 

\subsubsection{\texttt{Goto}-Programm}

Das \texttt{Goto}-Programm ist relativ simpel. Man speichert sich die Werte der
beiden Zahlen, prüft, ob die erste ungleich null ist, wenn nein, gibt man sie
zurück. Wenn ja, prüft man, ob die zweite Zahl ungleich null ist. Wenn
nein, gibt man die zweite Zahl zurück (kleiner als 0 kann das Minimum nicht sein).
Wenn ja, so dekrementiert man beide
Zahlen (bzw. ihre Kopien) und beginnt wieder oben.

\begin{lstlisting}[escapeinside={b}{b}]
    xb$_4$b := xb$_1$b
    xb$_5$b := xb$_2$b
Lb$_1$b: if(xb$_4 \neq $b0) goto Lb$_2$b
    xb$_3$b := xb$_1$b
    halt
Lb$_2$b: if(xb$_5 \neq $b0) goto Lb$_3$b
    xb$_3$b := xb$_2$b
    halt
Lb$_3$b: goto Lb$_4$b
Lb$_4$b: xb$_4$b := xb$_4$b - 1
    xb$_5$b := xb$_5$b - 1
    goto Lb$_1$b

\end{lstlisting}

\subsubsection{\texttt{While}-Programm}

Aus dem obigen Programm wird gemäß dem Vorgehen aus den Slides ein
\texttt{While}-Programm generiert. Dies wird verkompliziert dadurch, dass
Prüfungen der Form \texttt{if(x = $c$)} nicht erlaubt sind. Daher wird eine
Variable immer dekrementiert, und geprüft, ob sie 0 ist. So stellt man fest,
welchen Wert der Programmzeiger ($x_{pos}$) hat. Ärgerlicherweise kann man nur
auf Ungleichheit prüfen, sodass die Anweisungen für den Fall, dass der Zeiger
einen bestimmten Wert hatte, in die \texttt{else}-Klauseln am Ende wandern. Das
Ganze muss geschatelt werden und es kommt folgender Programmsalat dabei heraus:

\begin{lstlisting}[escapeinside={[}{]}]
x[$_0$] := 1; x[$_4$] := x[$_1$]; x[$_5$] := x[$_2$]
while(x[$_0$] [$\neq$] 0) {
   x[$_6$] := x[$_0$] - 1 // test if pos is 1
   if(x[$_6$] [$\neq$] 0) {
      x[$_6$] := x[$_6$] - 1 // test if pos is 2
      if(x[$_6$] [$\neq$] 0) {
         x[$_6$] := x[$_6$] - 1 // test if pos is 3
         if(x[$_6$] [$\neq$] 0) {
            x[$_0$] := 1 // jump to L[$_1$]
         } else { // pos is 4
            x[$_4$] := x[$_4$] - 1
            x[$_5$] := x[$_5$] - 1
            x[$_0$] := 1 // start at top
         }
      } else { // pos was 2
         if(x[$_5$] [$\neq$] 0) { x[$_0$] := 3 } // jump to L[$_3$]
         else {
            x[$_3$] := x[$_2$] // x[$_5$] is 0, so the first one is at least as large
            x[$_0$] := 0 // terminate
         }
      }
   } else { // pos was 1
      if(x[$_4$] [$\neq$] 0) { x[$_0$] := 2 } 
      else {
         x[$_3$] := x[$_1$]; // if x[$_4$] is 0, the other one is at least as large
         x[$_0$] := 0 // terminate
      }
   }
}
\end{lstlisting}

Per Hand lässt sich eine einfachere Möglichkeit finden (Dank an Hendrik).

\begin{lstlisting}[escapeinside={[}{]}]
x[$_f$] := 1;
t[$_1$] := x[$_1$];
t[$_2$] := x[$_2$];
while(x[$_f$] != 0) {
   if(t[$_1$] != 0) {
      if(t[$_2$] != 0){
         t[$_1$] := t[$_1$] - 1;
         t[$_2$] := t[$_2$] - 1;
      } else {
         x[$_3$] := x[$_2$];
         x[$_f$] := 0;
      }
   } else {
      x[$_3$] := x[$_1$];
      x[$_f$] := 0;
   }
}
\end{lstlisting}

\subsection{} 

\subsection{} 

\begin{minipage}{.4\textwidth}
   \def\PuzzleSolutionContent#1{\makebox(1,1){\itshape{#1}}}
   \renewcommand{\PuzzleLineThickness}{1pt}
   \PuzzleSolution
   \begin{Puzzle}{4}{13}
   |a|*|*|*|*|*|c|*|a|*|*|*|*|.
   |c|*|*|*|*|*|b|*|c|*|*|*|*|.
   |b|*|*|*|*|a|a|c|a|c|*|*|*|.
   |a|c|b|c|b|c|a|a|c|a|c|a|b|.
   \end{Puzzle}
\end{minipage}

\end{document}
