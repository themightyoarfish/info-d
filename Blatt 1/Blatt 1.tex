\documentclass{article}
\title{Informatik \rotatebox[origin=c]{180}{D}\raisebox{2pt}{:} -- Blatt 1}
\author{Rasmus Diederichsen}
\date{\today}
\usepackage[ngerman]{babel}
\usepackage{microtype,
            xcolor,
            infodcw,
            tikzsymbols,
            IEEEtrantools,
            array,
            amsmath,
            amssymb,
            graphicx, subcaption,
            lmodern}
\usepackage[pdftitle={Informatik :D -- Blatt 1}, 
       pdfauthor={Rasmus Diederichsen}, 
       hyperfootnotes=true,
       colorlinks,
       bookmarksnumbered = true,
       linkcolor = blue,
       plainpages = false,
       citecolor = blue]{hyperref}
\usepackage[T1]{fontenc}
\usepackage[all]{hypcap}
\renewcommand{\thesection}{}
\renewcommand{\thesubsection}{Aufgabe \arabic{section}.\arabic{subsection}}
\renewcommand{\thesubsubsection}{}

\begin{document}

   \maketitle

   \section{} 
   \subsection{} 

   Die Reihenfolge ist 
   \begin{IEEEeqnarray*}{CCCCCCCC}
      & \mathcal{O}\left(\log x\right) & \subseteq & \textcolor{teal}{\mathcal{O}\left(\sqrt{x}\right)} & \textcolor{teal}{\subseteq} &
      \textcolor{teal}{\mathcal{O}\left(\frac{x}{\sqrt{x}}\right)} & \subseteq &
      \mathcal{O}\left(n^{0.99}\right) \\ 
      \subseteq & \textcolor{teal}{\mathcal{O}\left(\frac{x}{\log{x}}\right)} &
      \textcolor{teal}{\subseteq} &
      \textcolor{teal}{\mathcal{O}\left(\frac{x}{\log{\log{x}}}\right)} &
      \subseteq & \mathcal{O}\left(x\log{x}\right) &
      \subseteq & \mathcal{O}\left(x\log{x^2}\right) \\
      \subseteq & \mathcal{O}\left(x\log^2 x\right) & 
      \subseteq & \mathcal{O}\left(x^{1.01}\right) & \subseteq &
      \textcolor{teal}{\mathcal{O}\left(2^x\right)} & \textcolor{teal}{\subseteq} &
      \textcolor{teal}{\mathcal{O}\left(2^{x+1}\right)} \\
      \subseteq & \mathcal{O}\left(2^{2x}\right)
   \end{IEEEeqnarray*}

   Dies kann leichter nachvollzogen werden, f\"uhrt man sich vor Augen, dass
   $\log{x}$ asymptotisch langsamer w\"achst als jede Potenz von $x$ und damit
   jedes Polynom in $x$ (ohne Beweis). Weiterhin w\"achst jede
   Exponentialfunktion asymptotisch st\"arker als jedes Polynom (ebenfalls hier
   ohne Beweis). 
   
   \subsection{} 

   \subsubsection{I}

   Zu Zeigen:
   \begin{equation*}
      f_2(n) = n^2 + 1000n \in \mathcal{O}\left(n^2\right)
   \end{equation*}
   Wir betrachten $n>0$.
   \begin{IEEEeqnarray*}{rCl}
      n^2 + 1000n & \le & cn^2 \\
      n(n + 1000) & \le & cn^2 \\
      n + 1000 & \le & cn
   \end{IEEEeqnarray*}
   Sei nun $c = 1001$. F\"ur alle $n> 0$ gilt
   \begin{IEEEeqnarray*}{rCl}
      n + 1000 & \le & 1001n \\
      1000 & \le & 1000n
   \end{IEEEeqnarray*}
   Daher gibt es f\"ur alle $n>0$ ein $c$ sodass $f_1(n)\le cf_2(n)$, dies gilt
   f\"ur alle $c \ge 1001$.

   \subsubsection{II}

   Zu Zeigen:
   \begin{equation*}
      f_3(n) = m^{\log n} \le cn^{\log m}
   \end{equation*}
   Es ist
   \begin{IEEEeqnarray*}{rCl}
      m^{\log n} & \le & cn^{\log m} \\
      \log{m^{\log n}} & \le & \log{cn^{\log m}} \\
      \log{n}\log{m} & \le & \log{n}\log{m} + \log{c}\log{m} \\
      \log{n} & \le & \log{n} + \log{c}
   \end{IEEEeqnarray*}
   Die letzte Ungleichung ist erf\"ullt f\"ur $c\ge1$.
   
   \subsubsection{III}
   
   Zu Zeigen:
   \begin{equation*}
      f_6(n) \not\in f_5(n)
   \end{equation*}
   Die Beziehung ist erf\"ullt genau dann wenn $\forall c \forall n_0 > 0: \exists
   n \ge n_0: f_6(n) > cf_5(n)$. Sei daher $c$ beliebig fest gew\"ahlt .
   Die Ungleichung ist genau dann l\"osbar, wenn $n$ ungerade ist und $> 100$. In
   diesem Fall
   \begin{IEEEeqnarray*}{rCl}
      f_6(n) & > & cf_5(n) \\
      n^3 & > & cn \\
      n & > & \sqrt{c}
   \end{IEEEeqnarray*}

   Die Wahl von $n_0$ ist hier nicht relevant, da ich f\"ur jedes beliebige $c$
   ein ungerades und beliebig gro\ss{}es $n > n_0$ finden kann, sodass $n >
   \sqrt{c}$ gilt.

   \subsection{} 

   \subsubsection{a)}

   \begin{equation*}
      L_{min} = \{aa, ab, ac, ca, ba\}
   \end{equation*}
   \begin{equation*}
      L_{max} = \{ab, ba, ac, ca, bb, cc, bc, cb\}
   \end{equation*}
   \begin{equation*}
      L_{min} \cap L_{max} = \{ab, ac, ca, ba\}
   \end{equation*}
   \begin{equation*}
      L_{min} \cup L_{max} = \{aa, ab, ba, ac, ca, bb, cc, bc, cb\}
   \end{equation*}
   \begin{equation*}
      L_{min} \setminus L_{max} = \{aa\}
   \end{equation*}
   \subsubsection{b)}

   \begin{eqnarray*}
      \overline{L_{min}} & = &  \Sigma^* \setminus L_{min} \\
                         & = & \{w~ |~ w \in \Sigma^*, w \not\in L_{min}\} \\
                         & = & \{a, b, c, bb, cc, cb, bc\} \cup \{ w \in
                               \Sigma^* \mid |w| > 2\} \cup \{\varepsilon\}
   \end{eqnarray*}
   \subsubsection{c)}
   
   \begin{equation*}
      \left(L_{min}\setminus L_{max}\right)^* = \{a^{2i} \mid i \in
      \mathbb{N}\}
   \end{equation*}

   \subsection{} 
   
   \subsubsection{a)}
   
   \begin{equation*}
      \{a^i \mid i > 0, i \kern-8pt\mod 2 = 1\} \cup \{b^i \mid i > 1, i
      \kern-8pt\mod = 0\}
   \end{equation*}

   \subsubsection{b)}

   \begin{equation*}
      \{b(ca)^i \mid i \ge 0\} \cup \{a(ca)^i \mid i \ge 0\}
   \end{equation*}

   \subsubsection{c)}

   \begin{equation*}
      \{aba, bab, abb\}^+
   \end{equation*}

   \subsection{} 
   
   \begin{center}
      \infodcwpuzzle{0/0/c,1/0/c,2/0/c,3/0/d,4/0/d,5/0/b,0/1/c,3/1/a,5/1/d,0/2/b,
      1/2/d,2/2/d,3/2/b,5/2/c,0/3/b,1/3/d,0/4/a,1/4/c,0/5/a,1/5/c,2/5/b,3/5/c,4/5/b,5/5/a}{1/0/5/bottom,2/0/5/right,3/1/5/bottom,4/0/2/right,5/0/0/right,6/3/2/bottom,7/5/2/bottom}
   \end{center}
   % \def\PuzzleSolutionContent#1{\makebox(1,1){\itshape{#1}}}
   % \renewcommand{\PuzzleLineThickness}{1pt}
   % \PuzzleSolution
   % \begin{Puzzle}{6}{6}
   %    |a |c |b |c |b |a|.
   %    |a|c|*|*|*|*|.
   %    |b|d|*|*|*|*|.
   %    |b|d|d|b|[][.]*|c|.
   %    |c|[][.]*|[][.]*|a|[][.]*|d|.
   %    |c|c|c|d|d|b|.
   % \end{Puzzle}

   \subsection{} 
   
   \subsubsection{a)}
      
   \renewcommand{\r}{\rightarrow}
      \begin{IEEEeqnarray*}{rCl}
         S & \r & \Smiley A \\
         A & \r & \Smiley B \\
         B & \r & \Smiley B \mid \Neutrey B \mid \Sadey B \mid \Sadey C \\
         C & \r & \Sadey
      \end{IEEEeqnarray*}

   \subsubsection{b)}

   \begin{IEEEeqnarray*}{rCl}
      S & \r & \Smiley S \mid \Sadey S \mid \Neutrey S \mid \Neutrey A \\
      A & \r & \Neutrey B \\
      B & \r & \Smiley B \mid \Sadey B \mid \Neutrey B \mid \Neutrey C \\
      C & \r & \Neutrey D \\
      D & \r & \Smiley D \mid \Sadey D \mid \Neutrey D \mid \Neutrey E \\
      E & \r & \Neutrey F \\
      F & \r & \Smiley F \mid \Sadey F \mid \Neutrey F \mid \Smiley \mid
      \Neutrey \mid \Sadey
   \end{IEEEeqnarray*}

   \subsubsection{c)}

   \begin{IEEEeqnarray*}{rCl}
      S & \r & \Neutrey S \mid \Sadey S \mid \Smiley A \\
      A & \r & \Neutrey S \mid \Sadey S \mid \Smiley B \\
      B & \r & \Neutrey C \mid \Sadey C \\
      C & \r & \Neutrey C \mid \Sadey C \mid \Smiley D \\
      D & \r & \Neutrey C \mid \Sadey C \mid \Smiley E \mid \Smiley \\
      E & \r & \Neutrey \mid \Sadey \mid \Neutrey D \mid \Sadey D
   \end{IEEEeqnarray*}
\end{document}
