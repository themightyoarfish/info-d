\documentclass{scrartcl}
\title{Informatik D Klausur SS2014 -- Gruppe Reis}
\author{Rasmus Diederichsen \and Sebastian H\"offner}
\date{Letzte \"Anderung: \today}
\usepackage{microtype,
   enumitem,
   calligra,
   lastpage,
   titlesec,
   listings,
   xcolor,
   IEEEtrantools,
   tabularx,
   tikz,
   fancyhdr,
   array,
   amsmath,
   amssymb,
   graphicx,
   colortbl,
   subcaption,
   lmodern}
\usetikzlibrary{automata,arrows}
\tikzstyle{every state}=[ minimum size=.8cm ]
\usepackage[ngerman]{babel}
\usepackage[top=2cm,bottom=2.5cm,right=2cm,left=2cm]{geometry}
\usepackage[pdftitle={Info D Klausur SS2014 Gruppe Reis}, 
   pdfauthor={Rasmus Diederichsen}, 
   hyperfootnotes=true,
   colorlinks,
   bookmarksnumbered = true,
   linkcolor = black,
   plainpages = false,
   citecolor = lightgray]{hyperref}
\lstset{
   basicstyle=\footnotesize\ttfamily,
   language=Python,
   numbersep=-5pt,
   breaklines=true,
   commentstyle=\color{blue},
   keywordstyle=\color{purple}\textbf,
   numberstyle=\tiny\color{gray},
   numbers=left,
   stringstyle=\color{olive},
}
\usepackage[utf8]{inputenc}
\usepackage[T1]{fontenc}
\usepackage[all]{hypcap}
\titleformat{\subsection}{\normalfont\fontsize{12}{15}\bfseries}{\thesubsection}{1em}{}
\titleformat{\section}{\normalfont\fontsize{12}{15}\bfseries}{\thesection}{1em}{}
\DeclareMathOperator*{\op}{\bigcirc}
\renewcommand{\headrulewidth}{0pt}
\renewcommand{\footrulewidth}{0pt}
% for last page footer
\fancypagestyle{lastpage}{
   \fancyhf{}
   \fancyfoot[L]{\emph{Blatt \thepage{} von \pageref{LastPage}}}
   \fancyfoot[R]{\calligra Viel Erfolg!}
}

\pagestyle{fancy}
\fancyhf{}
\fancyfoot[L]{\emph{Blatt \thepage{} von \pageref{LastPage}}}

\setlength{\fboxsep}{10pt} % padding for boxes
\setlength{\parindent}{0pt}

\begin{document}

\begin{titlepage}
   \maketitle

   \begin{abstract}
      Dies sind unsere Lösungsvorschläge zur Reis-Klausur von 2014, die wir
      zusammengestellt haben, nachdem wir unsere korrigierten Klausuren
      eingesehen haben. Die Lösungen sollten also mehr oder weniger stimmen. Mit
      Hilfe dieser Lösungen kann man auch die meisten Fragen der zweiten Gruppe
      beantworten. Bitte teilt uns Korrekturbedarf über die üblichen Kanäle mit.
   \end{abstract}
\end{titlepage}

\section*{Aufgabe 1: Sprache vs. Grammatik \hfill (12 Punkte)} 

\subsection*{(a) Hierarchie und Automaten \hfill \normalfont (10 Punkte)}
Zu jeder Sprache gibt es entsprechende Automaten. Vervollständigen Sie die 
folgende Tabelle:
{
   % this cell content alignment shit is credited to
   % http://tex.stackexchange.com/a/195604/31250
   \newlength\origtabcolsep
   \origtabcolsep=\tabcolsep
   \tabcolsep=0pt
   \renewcommand{\tabularxcolumn}[1]{m{#1}}
   \newcolumntype{e}{>{\hsize=0pt}X}
   \newcolumntype{x}{>{\hskip\origtabcolsep}X<{\hskip\origtabcolsep}}
   \newcolumntype{C}{>{\hskip\origtabcolsep\hspace{1em}}l<{\hskip\origtabcolsep}}
   \begin{center}
      \begin{tabularx}{\textwidth}{|C|x|xe|}
         \hline
         Chomsky-Typ & Name der Sprachfamilie & Automaten & \\
         \hline\hline\hline
         0 & \cellcolor{lightgray!20}rekursiv aufzählbar & \cellcolor{lightgray!20}(N)DTM & \\[3em]
         \hline
         \cellcolor{lightgray!20}1 & \cellcolor{lightgray!20}kontext-sensitiv & NLBA & \\[3em]
         \hline
         \cellcolor{lightgray!20}3 & regulär & \cellcolor{lightgray!20}(N)DEA & \\[3em]
         \hline
      \end{tabularx}
   \end{center}
}


\subsection*{(b) Grammatikdefinition \hfill{} \normalfont{(2 Punkte)}} 
Definieren Sie kontextfreie Grammatiken (für Sprachen $L$ mit
\vspace{\baselineskip}
$\varepsilon\not\in L$).

\begin{center}
   \noindent\fcolorbox{black}{lightgray!20}{
      \begin{minipage}[t][3.5cm]{.95\textwidth}
         \emph{Alle Regeln haben die Form\ldots}
         \vspace{\baselineskip}
         \begin{equation*}
            V \rightarrow (V \cup \Sigma)^+
         \end{equation*}
         \vspace{\baselineskip}
      \end{minipage}
   }
\end{center}

\section*{Aufgabe 2: Sprachen \hfill{} (12 Punkte)}

\newcommand{\mpsol}{\setlength{\fboxsep}{3pt}\colorbox{lightgray!20}{$\boxtimes$}}
\renewcommand{\mp}{\setlength{\fboxsep}{3pt}\colorbox{lightgray!20}{$\square$}}

{\renewcommand{\arraystretch}{1.4}
   \begin{tabularx}{\textwidth}{ccX}
      korrekt & falsch & \\ \hline
      \mpsol & \mp & Die Sprache $\left\{0^k1^\ell2^k22^\ell \mid k,\ell\ge 0\right\}$ ist kontextfrei.\\
      \mpsol & \mp & Nicht-deterministische Kellerautomaten, die maximal ein Symbol im Keller
      speichern können, sind nur so mächtig wie deterministische
      endliche Automaten. \\
      \mp & \mpsol & Man kann jeden deterministischen Kellerautomaten in einen
      nichtdeterministischen Kellerautomaten umformen, dessen Keller immer maximal 2
      Elemente enthält.\\
      \mp & \mpsol & Es gibt endliche Mengen, die nicht das Alphabet einer Sprache
      sein können. \\
      \mpsol & \mp & Das Alphabet einer Sprache kann unendlich groß sein.\\
      \mp & \mpsol & $\Sigma^*$ ist die Potenzmenge von $\Sigma$.
\end{tabularx}}

\pagebreak
\section*{Aufgabe 3: Pumping Lemma \hfill (16 Punkte)}
\subsection*{(a) Definition \hfill \normalfont (4 Punkte)}
Wie lautet das Pumping Lemma für reguläre Sprachen?

\vspace{4pt}
\noindent\fcolorbox{black}{lightgray!20}{
   \begin{minipage}[c][6cm]{.95\textwidth}
      Sei $L$ eine reguläre Sprache. Es gibt eine Zahl $n:=n(L)$, sodass alle 
      Wörter $z \in L$ mit $|z| \ge n$ sich zerlegen lassen als $z = uvw$ mit den 
      Eigenschaften $|v| \ge 1$, $|uv| \le n$, $uv^*w \in L$.
   \end{minipage}
}

\subsection*{(b) Anwendung \hfill \normalfont (12 Punkte)}
Beweisen Sie, dass $\left\{c^ja^ica^{3i}c \mid i,j \ge 0\right\}$ keine 
reguläre Sprache ist.

\vspace{4pt}
\noindent\fcolorbox{black}{lightgray!20}{
   \begin{minipage}[c][13.6cm]{.95\textwidth}

      Betrachten wir das Wort $z=c^0a^nca^{3n}$. Dann sind die möglichen
      Zerlegungen $uv = a^{n-l}, l \ge 0$ und $w = a^lca^{3n}$.  In $v$ sind so
      garantiert nur $a$s, d.h. $z^\prime=uv^0w \notin L$ und $z^{\prime\prime}=uvv^+w \notin L$,
      da die Anzahl $a$s am vor dem trennenden $c$ nicht mehr zur Anzahl $a$s
      nach dem $c$ passt. 
      \hfill{}$\square$
   \end{minipage}
}

\pagebreak
\section*{Aufgabe 4: RegEx vs. DEA \hfill (12 Punkte)}
Geben Sie einen deterministischen endlichen Automaten an, der dem folgenden 
regulären Ausdruck entspricht:
\begin{align*}
   \left(\varnothing^*\middle|23^*2\middle|2^*\right)1
\end{align*}
(Es gibt genug Platz, damit Sie Zwischenschritte aufschreiben können. Markieren 
Sie Ihr Endergebnis bitte entsprechend.)

\vspace{4pt}
\noindent\fcolorbox{black}{lightgray!20}{
   \begin{minipage}[c][20cm]{.95\textwidth}
      \begin{center}
         \begin{tikzpicture}[->, auto, node distance=3cm]
            \node[state,initial]   (1)                    {};
            \node[state]           (2) [below of=1]       {};
            \node[state]           (3) [below right of=2] {};
            \node[state]           (4) [right of=3]       {};
            \node[state,accepting] (5) [above right of=4] {};
            \node[state]           (6) [below of=4]       {};
            \path (1) edge [bend left=20]         node {1} (5)
            edge                        node {2} (2)
            (2) edge                        node {1} (5)
            edge [bend right=40, left]  node {2} (6)
            edge                        node {3} (3)
            (3) edge                        node {2} (4)
            edge [loop below]           node {3} (3)
            (4) edge                        node {1} (5)
            (6) edge [bend right=10, right] node {1} (5)
            edge [loop below]           node {2} (6)
            ;
         \end{tikzpicture}
      \end{center}
   \end{minipage}
}

\pagebreak
\section*{Aufgabe 5: Kellerautomat \hfill (10 Punkte)}
Geben Sie für die Sprache $\left\{a^jcb^{j-i}cd^i\mid j \ge i \ge 0 \right\}$
einen Kellerautomaten an, der durch leeren Keller akzeptiert.

\vspace{4pt}
\noindent\fcolorbox{black}{lightgray!20}{
   \begin{minipage}[c][7cm]{.95\textwidth}
      \begin{center}
         \begin{tikzpicture}[->, auto, node distance=3cm,
            ]
            \node[state,initial]   (1)              {};
            \node[state]           (2) [right of=1, node distance=3cm] {};
            \node[state]           (3) [right of=2] {};
            \node[state]           (4) [below of=3] {};
            \path (1) edge [loop above] node {$a,\circ\!\in\!\Gamma/a\circ$} (1)
            edge              node {$c,\circ\!\in\!\Gamma/a\circ$} (2)
            (2) edge [loop above] node {$b,a/\epsilon$} (2)
            edge              node {$c,a/a$} (3)
            edge [left]       node {$c,\#/\epsilon$} (4)
            (3) edge [loop above] node {$d,a/\epsilon$} (3)
            edge [right]      node {$\epsilon,\#/\epsilon$} (4)
            ;
         \end{tikzpicture}
      \end{center}
   \end{minipage}
}

\section*{Aufgabe 6: Rechnende Turingmaschine \hfill (12 Punkte)}
Gegeben eine binär kodierte Zahl $\alpha$. Geben Sie eine Turingmaschine an,
die die folgende Funktion berechnet:
\begin{align*}
   f(\alpha):=\begin{cases}
      \text{undef} & \alpha \in \mathbb{N}_g \cup \{0\} \\
      \alpha - 1 & \text{sonst}
   \end{cases}
\end{align*}

\noindent\fcolorbox{black}{lightgray!20}{
   \begin{minipage}[c][10cm]{.95\textwidth}
      \begin{center}
         \begin{tikzpicture}[->, auto, node distance=3cm,
            ]
            \node[state,initial]   (1)              {};
            \node[state]           (2) [right of=1] {};
            \node[state]           (3) [right of=2] {};
            \node[state]           (4) [right of=3] {};
            \node[state]           (5) [below of=2] {};
            \path (1) edge [loop below] node[align=center] {$1/1,R$\\$0/0,R$} (1)
            edge              node {$\square/\square,L$} (2)
            (2) edge              node {$1/0,L$} (3)
            edge              node {$0/0,R$} (5)
            (3) edge [loop below] node[align=center] {$1/1,L$\\$0/0,L$} (3)
            edge              node {$\square/\square,R$} (4)
            (5) edge [loop below] node {$\square/\square,S$} (5)
            ;
         \end{tikzpicture}
      \end{center}
   \end{minipage}
}

\pagebreak
\section*{Aufgabe 7: LOOP-Berechenbarkeit \hfill (12 Punkte)}
\subsection*{(a) LOOP-Programm \hfill \normalfont (8 Punkte)}
Geben Sie ein LOOP-Programm an (eingeschränkte Definition, d.h. keine Addition
von Variablen oder höheren Rechenoperationen), dass der folgenden Codezeile
entspricht:
\begin{align*}
   x_4 := 3 \cdot x_1 \cdot x_2
\end{align*}

\newsavebox{\mybox}
\begin{lrbox}{\mybox}
   \begin{lstlisting}[escapeinside={*}{*}]
   x*$_4$* := 0
   loop(3) {
      loop(x*$_1$*) {
         loop(x*$_2$*) {
            x*$_4$* := x*$_4$* + 1
         }
      }
   }
   \end{lstlisting}
\end{lrbox}

\noindent\fcolorbox{black}{lightgray!20}{
   \begin{minipage}[c][8cm]{.95\textwidth}
      \begin{center}
         \usebox{\mybox}
      \end{center}
   \end{minipage}
}
\subsection*{(b) Turing-vollständig \hfill \normalfont (4 Punkte)}
Begründen Sie, warum LOOP-Programme nicht so mächtig wie Turingmaschinen sind.

\vspace{4pt}
\noindent\fcolorbox{black}{lightgray!20}{
   \begin{minipage}[c][5cm]{.95\textwidth}
      Da LOOP-Programme nur eine jeweils konstante Anzahl Schleifeniterationen
      durchführen können, gibt es Funktionen, die so nicht berechenbar sind.
      Eine bekannte Funktion ist die Ackermannfunktion.
   \end{minipage}
}

\pagebreak

\section*{Aufgabe 8: Entscheidbarkeit \hfill{} (12 Punkte)}

Wir definieren das Ergebnis des \emph{Klebeoperators} $\circ$ als die Zahl, die
durch das Hintereinanderschreiben der Dezimaldarstellungen ihrer einzelnen
Argumente repräentiert wird. Wir können mehrere Klebeoperationen gesammelt
schreiben, z.B.

\begin{equation*}
   \op\limits_{i=1}^4 i^2 = 1 \circ 4 \circ 9 \circ 16 = 14916.
\end{equation*}
Betrachten Sie das folgende Problem:
\begin{center}
   \fbox{
      \begin{minipage}{.95\textwidth}
         \textbf{Gegeben}: Drei jeweils $m-$elementige Mengen $\mathcal{A} :=
         \left\{a_i\right\}_{1\le i\le k}$, $\mathcal{B} :=
         \left\{b_i\right\}_{1\le i\le k}$, $\mathcal{C} :=
         \left\{c_i\right\}_{1\le i\le k}$ mit Elementen aus $\mathbb{N}$.

         \noindent\textbf{Frage}: Gibt es eine Sequenz $s(1),s(2),\ldots,s(n)$ mit $n \ge
         1$ und $s(i) \in \{1,\ldots,k\}$ für alle $1 \le i \le n$, sodass 

         \begin{tabularx}{\textwidth}{>{\raggedleft\arraybackslash}p{.5\textwidth}cl>{\raggedleft\arraybackslash}X}
            % \begin{IEEEeqnarray*}{+rCl+r*}
            $\op\limits_{i=1}^na_{s(i)} - \op\limits_{i=1}^nb_{s(i)}$ & = & $\op\limits_{i=1}^nc_{s(i)}$ &  ("`Gleichung"')
         \end{tabularx}
      \end{minipage}
   }
\end{center}

\subsection*{(a) Beispiele \hfill{} \normalfont{(4 Punkte)}}

Geben Sie jeweils ein Beispiel einer Ja- und einer Nein-Instanz für dieses
Problem an:\\[.5\baselineskip]
Ja-Instanz \hfill{} Nein-Instanz

\vspace{2pt}
\noindent\fcolorbox{black}{lightgray!20}{
   \begin{minipage}[c][3cm]{.40\textwidth}
      Sei $\mathcal{A} = \mathcal{B} := \left\{1,1,1\right\}, \mathcal{C}
      := \{0,0,0\}$. Offensichtlich ist die Instanz lösbar mit $s(1) =
      s(2) = s(3) \in \{1,2,3\}$.
   \end{minipage}
}
\hfill{}\fcolorbox{black}{lightgray!20}{
   \begin{minipage}[c][3cm]{.40\textwidth}
      $\mathcal{A} = \mathcal{B} = \mathcal{C} := \left\{1,2,3\right\}$
   \end{minipage}
}

\subsection*{(b) Semi-Entscheidbarkeit \hfill{} \normalfont{(4 Punkte)}}

% lstlisting cannot go inside fbox, so we need to do this shit
\newsavebox{\codebox}
\begin{lrbox}{\codebox}
   \begin{lstlisting}[escapeinside={*}{*}]
   for n = 1,...,*$\infty$*:
      foreach *$\vec{i}$* in *$\{1,...,n\}^k$*:
         if *$\op\limits_{i=1}^na_{\vec{i}[i]} - \op\limits_{i=1}^nb_{\vec{i}[i]}$ = $\op\limits_{i=1}^nc_{\vec{i}[i]}$*:
            return true
   \end{lstlisting}
\end{lrbox}

\noindent\fcolorbox{black}{lightgray!20}{
   \begin{minipage}[c][9cm]{.95\textwidth}
      Wir beweisen die Semi-Entscheidbarkeit, indem wir einen Algorithmus
      angeben, der eine Lösung findet, sofern sie existiert.

      \begin{center}
         \usebox{\codebox}
      \end{center}

      Es gibt in einer $n$-Elementigen Menge nur endlich viele Auswahlen von $k$
      Elementen. Gibt es eine Lösung, wird sie also irgendwann gefunden.
   \end{minipage}
}

\subsection*{(c) Unentscheidbarkeit \hfill{} \normalfont{(4 Punkte)}}

Beschreiben Sie kurz die notwendige Reduktion (von? nach? wie?) um zu begründen,
warum das Problem nicht entscheidbar ist:

\vspace{4pt}
\noindent\fcolorbox{black}{lightgray!20}{
   \begin{minipage}[c][9cm]{.95\textwidth}
      Es muss ein unentscheidbares Problem auf das \emph{Klebeproblem} reduziert
      werden. Es bietet sich hierfür das PCP an. Gegeben eine PCP-Instanz mit
      Tupelmenge 
      \begin{equation*}
         \mathcal{M} = \{(x_i,y_i) \mid 1\le i\le k, x,y\in\mathbb{N}\}
      \end{equation*} und $|\mathcal{M}| = k$, erstellen wir eine Klebeinstanz
      mit 
      \begin{equation*}
         \mathcal{A} := \{x_i \mid 1\le i\le k,
         (x_i,y_i)\in\mathcal{M}\},~\mathcal{B} := \{0\}^k,~\mathcal{C} := \{y_i \mid
         1\le i \le k, (x_i,y_i)\in\mathcal{M}\}.
      \end{equation*} 
      Offensichtlich ist die PCP-Instanz lösbar, genau
      dann, wenn die Klebeinstanz lösbar ist, da
      \begin{equation*}
         \op\limits_{i=1}^ka_{s(i)} - \{0\}^k = \op\limits_{i=1}^kc_{s(i)}
      \end{equation*}
      gleichbedeutend ist mit der Aussage, dass links dieselbe Zahl steht wie
      rechts. Da links die $x_i$ der PCP-Instanz stehen und rechts die $y_i$,
      sind die Problemstellungen identisch. Man muss eine Auswahl an Indizes
      finden (wobei nicht alle $k$ vorkommen müssen, und Indizes mehrfach
      verwended werden können), sodass die konkatenierten $x_i$ gleich den
      konkatenierten $y_i$ sind, was genau der Klebeoperation entspricht.
   \end{minipage}
}

\section*{Aufgabe 9: {\sffamily \textbf{\itshape P}} vs. {\sffamily \textbf{\itshape NP}} \hfill{}(20 Punkte)}

\subsection*{(a) Definition \hfill{} \normalfont{(4 Punkte)}}

Definieren Sie die Komplexitätsklasse \textbf{\sffamily\itshape{P}}.

\vspace{4pt}
\noindent\fcolorbox{black}{lightgray!20}{
   \begin{minipage}[c][5cm]{.95\textwidth}
      \textbf{\itshape\sffamily P} ist die Menge aller Entscheidungsprobleme,
      die sich von einer deterministischen Turingmaschine in polynomieller Zeit
      entscheiden lassen. Alternativ ist es die Menge aller Probleme, für die
      ein Zeuge von einer deterministischen Turingmaschine in polynomieller Zeit
      verifizierbar ist.
   \end{minipage}
}

\pagebreak
\subsection*{(b) Basiszusammenhänge \hfill \normalfont (10 Punkte)}
Welche Aussagen stimmen?

\noindent\emph{(Achtung: Pro Frage gibt es +2/0/-2 Punkte bei einer richtigen/
   keinen/falschen Antwort! Sie erhalten jedoch natürlich mindestens 0 Punkte
für die gesamte Aufgabe.)}

\newcommand{\NP}{\textbf{\itshape\sffamily NP}}
\renewcommand{\P}{\textbf{\itshape\sffamily P}}
\vspace{1em}
% somehow the boxes get quite huge here
{\renewcommand{\arraystretch}{1.4}
   \begin{tabularx}{0.95\textwidth}{ccX}
      korrekt & falsch & \\ \hline
      \mp & \mpsol & Das Problem "`Finde das kleinste Element aus $n$ gegebenen 
      Zahlen"' liegt in \NP.\\
      \mpsol & \mp & Das Problem "`Gegeben ein gewichteter Graph und Zahl $W$.
      Kann man einen aufspannenenden Baum finden, dessen 
      Gesamtgewicht maximal $W$ ist?"' liegt in \P.\\
      \mp & \mpsol & Das Problem "`Gegeben ein Graph $n \ge 8$ Knoten. Kann man 
      einen Hamiltonkreis finden, der maximal $n-3$ Kanten 
      enthält?"' ist \NP-vollständig.\\
      \mpsol & \mp & Wenn ein stark \NP-vollständiges Problem einen
      pseudopolynomiellen Algorithmus erlaubt, gilt \NP=\P.\\
      \mpsol & \mp & Sei $\mathcal{A}$ das zu einem Optimierungsproblem 
      $\mathcal{B}$ zugehörige Entscheidungsproblem. Wenn 
      $\mathcal{A}\in$\P, gibt es einen 
      deterministischen polynomiellen Algorithmus für $\mathcal{B}$.
\end{tabularx}}

\subsection*{(c) Zeuge \hfill \normalfont (6 Punkte)}

Was versteht man, wenn man über \P{} und \NP{} spricht, unter einem \emph{Zeugen}?

\vspace{6pt}
\noindent\fcolorbox{black}{lightgray!20}{
   \begin{minipage}[t][5cm]{.95\textwidth}
      \emph{(Vollständige!) Definition:}

      \vspace{\baselineskip}
      Ein Zeuge ist eine Lösung für ein Entscheidungsproblem. Zur Lösung des
      Problems gehört die Überprüfung dieses Zeugens.
   \end{minipage}
}

\vspace{10pt}
Warum kann ein Zeuge nur polynomiell groß sein?

\vspace{6pt}
\noindent\fcolorbox{black}{lightgray!20}{
   \begin{minipage}[c][5cm]{.95\textwidth}
      Eine deterministische Turingmaschine kann in polynomieller Zeit nur
      polynomiell viele "`Elemente"' eines
      Zeugen betrachten. Ist dieser superpolynomiell groß, kann er nicht in
      polynomieller Zeit komplett betrachtet und daher auch nicht verifiziert
      werden.

   \end{minipage}
}

\pagebreak

\section*{Aufgabe 10: \NP-vollständig \hfill (12 Punkte)}

Sie kennen das Problem \textsc{Sat}, in dem eine Formel in konjunktiver Normalform
gegeben ist, und jede Klausel \emph{mindestens} ein Literal enthält. Sie kennen auch
den Spezialfall des 3-\textsc{Sat}, in dem jede Klausel \emph{maximal} drei Literale enthält.
Wir definieren nun das folgende Problem:


\vspace{4pt}
\noindent\fbox{
   \begin{minipage}{.95\textwidth}
      \begin{itemize}[itemsep=-2pt,leftmargin=3em,listparindent=.5cm,align=left]
         \item[\textbf{Problem:}]\textsc{Exakt-5-Sat} % funnily, label=\bfseries doe not do anything
         \item[\textbf{Gegeben:}] Eine aussagenlogische Formel $F$ in konjunktiver
            Normalform mit \emph{genau} fünf Literalen 

            pro Klausel. 
         \item[\textbf{Frage:}] Ist $F$ erfüllbar?
      \end{itemize}
   \end{minipage}
}

\vspace{\baselineskip}
Um zu zeigen, dass \textsc{Exakt-5-Sat} \NP-vollständig ist, zeigt man im Normalfall,
dass \ldots

\newsavebox\topalignbox
\newcolumntype{T}{%
   >{\begin{lrbox}\topalignbox
      \rule{0pt}{\ht\strutbox}}
      c
   <{\end{lrbox}%
      \raisebox{\dimexpr-\height+\ht\strutbox\relax}%
{\usebox\topalignbox}}}

\begin{center}
   \setlist[itemize]{itemsep=-3pt}
   \begin{tabular}{TTT}
      \setlength{\fboxsep}{2pt}
      \fbox{
         \begin{tabular}{l}
            \mp \ldots es vollständig in \P{} liegt \\
            \mp \ldots dass es in \P{} liegt\\
            \mpsol \ldots dass es in \NP{} liegt\\
            \mp \ldots dass es in \textbf{\sffamily\itshape NPI} liegt\\
            \hline 
            \emph{Punkt 1}
         \end{tabular}
      }
      & und & 
      \setlength{\fboxsep}{2pt}
      \fbox{
         \begin{tabular}{l}
            \mp \ldots es in \textbf{\sffamily\itshape CO-NP} liegt. \\
            \mp \ldots nicht \P{}-vollständig ist\\
            \mpsol \ldots \NP{}-schwer \\
            \mp \ldots nicht \NP{}-schwer \\
            \hline 
            \emph{Punkt 2}
         \end{tabular}
      }
   \end{tabular}
\end{center}

{\setlength\fboxsep{2pt}
   Punkt 1 ist trivial, daher beschränken wir uns auf Punkt 2. Dazu benötigen wir
   eine Reduktion \fbox{\mp über das\quad\mpsol von dem \quad\mp zu dem} Problem
   \fbox{\mpsol \textsc{3-Sat} \quad\mp \textsc{Sat} \quad\mp \textsc{Partition}} .
}

\vspace{3pt}
Geben Sie die notwendige Reduktion an, begründen Sie ihre notwendigen
Eigenschaften und beweisen Sie, dass der korrekte \emph{Punkt 2} gilt.

\vspace{4pt}
\noindent\fcolorbox{black}{lightgray!20}{
   \begin{minipage}[c][12.9cm]{.95\textwidth}
      Gegeben eine \textsc{3-Sat}-Instanz mit Klauselmenge $\mathcal{K}$,
      erstellt man eine \textsc{Exakt-5-Sat}-Instanz wie folgt:

      Für jede Klausel $K \in \mathcal{K}$ der Form $(\alpha_{1,k},\ldots,\alpha_{l,k})$,
      multipliziere das Literal $\alpha_k$ so oft, bis $K$ genau 5 Literale enthält.
      Aus beispielsweise 
      \begin{equation*}
         K_1 = (x_{1,1} \wedge \neg x_{2,1} \wedge x_{3,1}) 
      \end{equation*}
      wird so 
      \begin{equation*}
         K_1^\prime = (x_{1,1} \wedge \neg x_{2,1} \wedge x_{3,1} \wedge x_{3,1} \wedge x_{3,1}).
      \end{equation*} 
      Offensichtlich hat $K_1^\prime$ die selben
      Wahrheitsbedingungend wie $K_1$. Trivialerweise ist also die so kreierte
      \textsc{Exakt-5-Sat}-Instanz erfüllbar, wenn die \textsc{3-Sat}-Instanz
      erfüllbar ist. 
\end{minipage}}
\thispagestyle{lastpage}
\end{document}
