\documentclass{article}
\title{Informatik \rotatebox[origin=c]{180}{D}\raisebox{2pt}{:} -- Blatt 4}
\author{Rasmus Diederichsen \and Sebastian H\"offner}
\date{\today}
\usepackage[utf8]{inputenc}
\usepackage[ngerman]{babel}
\usepackage[svgnames]{xcolor}
\usepackage{wasysym}
\let\iiint\relax
\let\iint\relax
\usepackage{microtype,
            booktabs,
            mathtools,
            tikz,
            multicol,
            IEEEtrantools,
            array,
            amsmath,
            amssymb,
            graphicx, 
            lmodern}
\usepackage[pdftitle={Informatik D -- Blatt 4}, 
       pdfauthor={Rasmus Diederichsen}, 
       hyperfootnotes=true,
       colorlinks,
       bookmarksnumbered = true,
       linkcolor = blue,
       plainpages = false,
       citecolor = blue]{hyperref}
\usepackage[T1]{fontenc}
\usepackage[all]{hypcap}
\renewcommand{\thesection}{}
\renewcommand{\thesubsection}{Aufgabe \arabic{section}.\arabic{subsection}}
\renewcommand{\thesubsubsection}{}
\setcounter{section}{3}

\begin{document}

   \maketitle

   \section{} 
   \subsection{} 
   
   

   \usetikzlibrary{arrows,automata}
   \tikzset{every state/.style={fill=teal!20, line width=1pt, font=\sf}}
   \tikzset{every node/.style={font=\sf}}
   \tikzset{accepting/.append style={fill=blue!10}}

   % doesn't center, what the fuck
   \hspace{-2cm}\begin{tikzpicture}[node distance=7em,auto,
      thick,>=stealth',initial text={\sffamily start}]
      \node[state,initial] (A) {A};
      \node[state] (B) [below of=A] {B};
      \node[state] (C) [right of=A] {C};
      \node[state] (D) [right of=C] {D};
      \node[state] (F) [right of=D] {F};
      \node[state] (E) [above of=D] {E};
      \node[state] (E) [above of=D] {E};
      \node[state,accepting] (G) [below of=D] {G};

      \path[->]
      (A) edge[bend left] node {0} (B)
      (B) edge[bend left] node[text width=2em,pos=.4,shift={(.7em:.7em)}] {0:\textcolor{teal}{0},
                                                         1:\textcolor{teal}{1}} (A)
      (A) edge node {1} (C)
      (A) edge node {\#} (G)
      (C) edge[bend right=50] node[above] {0:\textcolor{teal}{1}} (A)
      (C) edge node {1:\textcolor{teal}{0}} (D)
      (D) edge[bend left] node {1} (E)
      (E) edge[bend left] node[text width=2em] 
         {0:\textcolor{teal}{0},1:\textcolor{teal}{1}} (D)
      (D) edge[bend left] node {0} (F)
      (F) edge[bend left] node {1:\textcolor{teal}{0}} (D)
      (F) edge[bend left=80,out=90,in=40,looseness=2] node {0:\textcolor{teal}{1}} (A)
      (D) edge node {\#:\textcolor{teal}{1}} (G);
   \end{tikzpicture}

   \subsection{} 
   
   Die Sprache $L$ ist definiert \"uber dem Alphabet $\Sigma = \{\sigma_1,
      \ldots, \sigma_k, *, (, ), |, \textcolor{teal}{\emptyset}\}$.
   Die Grammatik, die n\"otig ist, $L$ zu beschreiben, muss mindestens
   kontextfrei sein, denn mit regul\"aren Sprachen ist die Anforderung, dass jede
   ge\"offnete Klammer wieder geschlossen werden muss, nicht sicher zu stellen.

   Die Grammatik k\"onnte so aussehen:

   \begin{eqnarray*}
      S & \rightarrow & \textcolor{teal}{\sigma_1} \mid \ldots \mid
      \textcolor{teal}{\sigma_k} \\ %\mid \textcolor{teal}{(}S\textcolor{teal}{)}\\
      S & \rightarrow & \textcolor{teal}{(} S S \textcolor{teal}{)} \\
      S & \rightarrow & \textcolor{teal}{(} S \textcolor{teal}{\mid} S \textcolor{teal}{)} \\
      S & \rightarrow &
      \textcolor{teal}{(}S\textcolor{teal}{)}\textsuperscript{\textcolor{teal}{*}}\\
      ( S & \rightarrow & \emptyset )
   \end{eqnarray*}
   
   \emph{Anmerkung:} Es ist streitbar, inwiefern $\emptyset$ zum Alphabet
   geh\"ort, in jedem Fall ist $\emptyset\in L$. 

   \subsection{} 
   
   \subsubsection{a)}

   \begin{center}
   $L_1 = \{\eighthnote^i\halfnote^{2i} \mid i \ge 0\}$
   \end{center}

   Man nehme an, das Pumping Lemma sei durch $L_1$ erf\"ullt. Sei $n$ die
   Wortmindestl\"ange und $z =\eighthnote^n\halfnote^{2n} = uvw$ eine Zerlegung f\"ur $i = n$ mit $|z| \ge
   n$. Es gilt
   \begin{equation*}
      | uv | \le n
   \end{equation*}
   und 
   \begin{equation*}
      | v| = l \ge 1
   \end{equation*}
   also besteht $v$ nur aus $l$ Zeichen $\eighthnote$. Sei $|u|=k$.
   Da $uv^*w \in L_1$ ist $uv^{3n}w = \eighthnote^k\eighthnote^{3n}\halfnote^{2n} \in L_1$.
   Dies ist ein Widerspruch.

   \subsubsection{b)}
   
   \begin{center}
      $L_2 = \{\eighthnote^i\halfnote^j \mid 0 \le i \le j\}$
   \end{center}

   Man nehme an, das Pumping Lemma gilt. Sei $n$ die Wortmindestl\"ange, wir
   betrachten den Fall $i = n$. Sei $uvw = \eighthnote^n\halfnote^j$ die
   Zerlegung f\"ur ein beliebiges Wort $z$ mit $|z|\ge 2n$. Es gilt
   \begin{equation*}
      uv = \eighthnote^k, k \le n
   \end{equation*}
   Da $|v| \ge 1$ ist, gilt
   \begin{equation*}
      v = \eighthnote^l, l \ge 1
   \end{equation*}
   Da $uv^*w \in L_2$, ist $uv^{(j+1)l}w =
   \eighthnote^{k-l}\eighthnote^{(j+1)l}\halfnote^j\in L_2$. Dies ist ein
   Widerspruch.

   \subsubsection{c)}

   \begin{center}
      $L_3 = \{\eighthnote^{k^2} \mid k \ge 0\}$
   \end{center}

   Sei das Pumping Lemma erf\"ullt und $n$ die Wortmindestl\"ange. Wir
   betrachten den Fall $k = n$. Da $z =
   uvw \in L_3$ f\"ur $|z| \ge n$, ist $uv^2w \in L_3$. Es ist
   \begin{eqnarray*}
      |uv^2w| & = & |uvw| + |v| \stackrel{?}{<} (n+1)^2 \\
      n^2 + |v| & \stackrel{?}{<} & n^2 +2n + 1 \\
      |v| & < & 2n+1
   \end{eqnarray*}
   die letzte Ungleichung gilt, da $|uv| \le n$, mithin $|v| \le n$. Also ist
   $uv^2w$ zu kurz, um $\in L_3$ sein zu können, was ein Widerspruch ist.

   \subsection{} 

   \subsubsection{a)}

   F\"ur jede regul\"are Sprache gibt es einen deterministischen endlichen
   Automaten $\mathcal{A}$ mit Zustandsmenge $\mathcal{Z}$, der die Worte
   dieser Sprache akzeptiert. Sei die Anzahl der Zust\"ande $n$. Beim Lesen
   von $n$ Zeichen durchl\"auft der Automat $n+1$ Zust\"ande, ergo $\exists
   Z_k \in \mathcal{Z}$, der zwei oder mehr mal durchlaufen wird. 

   Sei $u$ die
   Zeichenkette, die $\mathcal{A}$ beim Durchlaufen von
   $Z_{start},\ldots,Z_k$ liest. Sei $v$ die Zeichenkette die $\mathcal{A}$
   beim Durchlaufen von $Z_k,\ldots,Z_k$ liest und $w$ die Zeichenkette, die
   $\mathcal{A}$ beim Durchlaufen der Zust\"ande $Z_k,\ldots,Z_{end}$
   durchl\"auft. Offensichtlich ist $0 < |v| \le n$ und $uv^*w$ durch den
   Auromaten akzeptiert.

   Da die betrachtete Sprache nicht regul\"ar ist, was das alternative Lemma
   aber nicht zeigt, eignet sich das origniale Lemma offenbar besser f\"ur
   die Pr\"ufung, da es striktere Anforderungen an die Sprache stellt.

   \subsubsection{b)}

   Nimmt man das originale Pumping Lemma als gegeben an, ist dies trivial, da
   $|uv| \le n \Rightarrow |v| \le n$.

   \subsubsection{c)}

   \textbf{Beweis der Nicht-Regularit\"at}\\[\baselineskip]
   Sei $n$ die Wortmindestl\"ange und $z = a^nb^n$. Es ist $|uv| \le n$, also
   \begin{equation*}
      v = a^k, k \ge 1
   \end{equation*}
   Da $uvw \in L(\Sigma)$, muss $uw = a^{n-k}b^n\in L(\Sigma)$ sein. Dies ist
   ein Widerspruch.\\[\baselineskip]

   \noindent\textbf{Beweis des alternativen Pumping Lemmas}\\[\baselineskip]
   Sei $n$ die Wortmindestl\"ange. Falls $|z| = n$, w\"ahle $v = z$.
   Offensichtlich ist $uv^*w = v^* \in L$. Falls $k = |z| > n$, gibt es einen
   Index $i$, sodass $z[i]\neq z[i+1]$, da $z$ $a$s und $b$s in gleicher Zahl
   enth\"alt. Da $u$ beliebig lang sein kann, w\"ahle $u = z[0]\ldots z[i-1], 
   v = z[i]z[i+1], w = z[i+2]\ldots z[k]$. Offensichtlich ist $uv^*w \in L$.

   \subsection{} 

   Wir nehmen an, $L$ sei regulär. Wir schneiden $L$ mit der regulären Sprache
   ${c}{a}^*{b}^*$
   \begin{equation*}
      L^\prime = L \cap \{c\}\{a\}^*\{b\}^* = \{c a^i b^i \mid i \ge 0\}
   \end{equation*}
   Wir definieren 
   Homomorphismus $h$ mit $h(c) = \varepsilon$, $h(a) = a$, $h(b) = b$ und wenden
   diesen auf $L^\prime$ an:
   \begin{equation*}
      L^{\prime\prime} = h(L^\prime) = \{a^i b^i \mid i \ge 0\}
   \end{equation*}

   Offensichtlich ist $L^{\prime\prime}$ nicht regulär,  mithin auch nicht
   $L^\prime$ und $L$.

   \subsection{} 

   Wenn die Sprache regul\"ar ist, dann k\"onnen wir z.B. einen endlichen
   Automaten oder regul\"aren Ausdruck finden, der die Sprache $L =
   \left\{a^ib^j \mid i \geq j \geq 0, j \leq 2000 \right\}$ beschreibt.

   Die Sprache $L$ l\"asst sich wie folgt als regul\"arer Ausdruck
   darstellen:

   \begin{equation*}
      a^0a^*b^0 \mid a^1a^*b^1 \mid a^2a^*b^2 \mid \ldots \mid a^{2000}a^*b^{2000}
   \end{equation*}

   Wir k\"onnen die Sprache $L$ auch als NDEA darstellen. Angenommen
   die Bedingung f\"ur $j$ ist nicht $j \leq 2000$ sondern $j \leq 0$, $j\leq
   1$ oder $j \leq 2$ (aus Gr\"unden der Aufschreibbarkeit).

   Dann w\"aren ein g\"ultige Automaten:

   \textbf{$j \leq 0$:}

   \begin{center}
      \begin{tikzpicture}[->, auto, node distance=2cm]
         \node[initial,state]   (Z1)               {};
         \node[state,accepting] (Z2) [right of=Z1] {};

         \path (Z1) edge              node {$\epsilon$} (Z2)
         (Z2) edge [loop right] node {a}          (Z2);
      \end{tikzpicture}
   \end{center}

   \textbf{$j \leq 1$:}

   \begin{center}
      \begin{tikzpicture}[->, auto, node distance=2cm]
         \node[initial,state]   (Z1)               {};
         \node[state]           (Z2) [right of=Z1] {};
         \node[state,accepting] (Z3) [below of=Z2] {};
         \node[state]           (Z4) [right of=Z2] {};
         \node[state,accepting] (Z5) [right of=Z4] {};

         \path (Z1) edge              node {$\epsilon$} (Z2)
         edge              node {$\epsilon$} (Z3)
         (Z2) edge              node {a}          (Z4)
         (Z3) edge [loop above] node {a}          (Z3)
         (Z4) edge [loop above] node {a}          (Z4)
         edge              node {b}          (Z5);
      \end{tikzpicture}
   \end{center}

   \textbf{$j \leq 2$:}

   \begin{center}
      \begin{tikzpicture}[->, auto, node distance=1.8cm]
         \node[initial,state]    (Z1)               {};
         \node[state]            (Z2) [right of=Z1] {};
         \node[state,accepting]  (Z3) [below of=Z2] {};
         \node[state]            (Z4) [right of=Z2] {};
         \node[state,accepting]  (Z5) [right of=Z4] {};
         \node[state]            (Z6) [above of=Z2] {};
         \node[state]            (Z7) [right of=Z6] {};
         \node[state]            (Z8) [right of=Z7] {};
         \node[state]            (Z9) [right of=Z8] {};
         \node[state,accepting] (Z10) [right of=Z9] {};


         \path (Z1) edge              node {$\epsilon$} (Z2)
         edge              node {$\epsilon$} (Z3)
         edge              node {$\epsilon$} (Z6)
         (Z2) edge              node {a}          (Z4)
         (Z3) edge [loop above] node {a}          (Z3)
         (Z4) edge [loop above] node {a}          (Z4)
         edge              node {b}          (Z5)
         (Z6) edge              node {a}          (Z7) 
         (Z7) edge              node {a}          (Z8)
         (Z8) edge [loop above] node {a}          (Z8)
         edge              node {b}          (Z9) 
         (Z9) edge              node {b}          (Z10);
      \end{tikzpicture}
   \end{center}

   Einen Automaten f\"ur $j \ge 2000$ kann man nach dem selben Muster aufbauen.

   Die Sprache $L = \left\{a^ib^j \mid i \geq j \geq 0, j \leq 2000
   \right\}$ ist also regul\"ar.


   \end{document}
