\documentclass{article}
\title{Informatik \rotatebox[origin=c]{180}{D}\raisebox{2pt}{:} -- Blatt 3}
\author{Rasmus Diederichsen}
\date{\today}
\usepackage[ngerman]{babel}
\usepackage[svgnames]{xcolor}
\usepackage{microtype,
            enumitem,
            colortbl,
            booktabs,
            multicol,
            cwpuzzle,
            tikzsymbols,
            IEEEtrantools,
            array,
            amsmath,
            amssymb,
            graphicx, subcaption,
            lmodern}
\usepackage[pdftitle={Informatik D -- Blatt 3}, 
       pdfauthor={Rasmus Diederichsen}, 
       hyperfootnotes=true,
       colorlinks,
       bookmarksnumbered = true,
       linkcolor = blue,
       plainpages = false,
       citecolor = blue]{hyperref}
\usepackage[T1]{fontenc}
\usepackage[all]{hypcap}
\renewcommand{\thesection}{}
\renewcommand{\thesubsection}{Aufgabe \arabic{section}.\arabic{subsection}}
\renewcommand{\thesubsubsection}{}
\setcounter{section}{2}

\begin{document}

   \maketitle

   \section{} 
   \subsection{} 

   Sei $L = \{a_1, a_2, \ldots, a_n\}$ eine endliche Sprache, $n < \infty$. Definiere f\"ur alle $i
   \in [1,n]$ Regeln $S_{i,k}$, wobei $k = |a_i|$
   \begin{eqnarray*}
      S = S_{i,1} & \rightarrow & a_i[1] S_{i,2} \\
      S_{i,2} & \rightarrow & a_i[2] S_{i,3} \\
          & \vdots & \\
      S_{i,k} & \rightarrow & a_i[k]
   \end{eqnarray*}

   Durch diese Aufz\"ahlung k\"onnen alle Worte in $L$ generiert
   werden. Alle $S_{i,k}$ sind regul\"ar, daher sind alle endlichen Sprachen
   regul\"ar.

   \subsection{} 
   \subsubsection{a)}

   Die regul\"are Grammatik ist

   \begin{IEEEeqnarray*}{rCl}
      A_1 & \rightarrow & 1A_2 \mid 1 \\
      A_2 & \rightarrow & 1A_2 \mid 0A_3 \mid 1 \\
      A_3 & \rightarrow & 0A_1 \mid 1
   \end{IEEEeqnarray*}
   \subsubsection{b)}

   Die obige Grammatik als nichtdeterministischer endlicher Automat ausgedr\"uckt ist
   
   \usetikzlibrary{arrows,automata}
   \begin{center}
      \begin{tikzpicture}[node distance=5em,auto,
         semithick]

         \node[initial,state] (Z1)               {$Z_1$};
         \node[state]         (Z2) [right of=Z1] {$Z_2$};
         \node[state]         (Z3) [right of=Z2] {$Z_3$};
         \node[accepting,state] (ZE) [below of=Z2] {$Z_{end}$};

         \path[->] 
         (Z1) edge node {1} (Z2)
         (Z1) edge node {1} (ZE)
         (Z2) edge[loop above] node[right,pos=.8] {1} ()
         (Z2) edge node {0} (Z3)
         (Z2) edge node {1} (ZE)
         (Z3) edge[bend right=80] node[above] {0} (Z1)
         (Z3) edge node {1} (ZE);
      \end{tikzpicture}
   \end{center}

   \subsection{} 
   
   Ein \"aquivalenter deterministischer Automat gem\"a\ss{} der Vorlesung
   ist\\[\baselineskip]

   \usetikzlibrary{arrows,automata}
   \begin{center}
      \begin{tikzpicture}[node distance=7em,auto,semithick]

         \node[initial,state,accepting] (12) {1,2};
         \node[state] (23) [right of=12] {2,3};
         \node[state] (235) [right of=23] {2,3,5};
         \node[state] (34) [below of=12] {3,4};
         \node[state] (3) [below of=23] {3};
         \node[state,accepting] (346) [right of=3] {3,4,6};
         \node[state] (5) [below of=3] {5};
         \node[state] (46) [below of=34] {4,6};
         \node[state,accepting] (23456) [below of=346] {2,3,4,5,6};

         \path[->]
         (12) edge node {1} (23)
         (23) edge node {1} (235)
         (235) edge node {0} (346)
         (235) edge[bend left] node {1} (23456)
         (12) edge node {0} (34)
         (23) edge node {0} (3)
         (34) edge node {1} (5)
         (3) edge node {1} (5)
         (46) edge[bend left] node {1} (5)
         (5) edge[bend left] node {1} (46)
         (5) edge[bend left=10] node {0} (346)
         (23456) edge node {0} (346)
         (23456) edge[loop right] node {1} ()
         (346) edge[bend left=10] node {1} (5);
      \end{tikzpicture}
   \end{center}

   \subsection{} 
   
   Zun\"achst separieren wir den Ausdruck in Baumschreibweise

   % this we use to append a little circled number to each node. works ONLY
   % within a tikz environment.
   \newcommand{\s}[1]{node[right=4pt,below=8pt,shape=circle,
         draw,inner sep=1pt,color=blue,font=\tiny] (char) {#1}}

   \begin{center}
      \begin{tikzpicture}[grow=right,level 1/.style={sibling distance=2cm},
                          level 2/.style={sibling distance=2cm}]
      \node {$((b\mid a)^* d\mid c)$}
         child { node {$(b\mid a)^*d$}
            child { node {$(b\mid a)^*$} child { node {$b\mid a$} 
                                                   child { node {$b$}\s{1}}
                                                   child { node {$a$}\s{2}} \s{5} }
                     \s{6}
                  }
                  child { node {$d$} \s{4} }
               \s{7}
         }
         child { node {$c$} \s{3} }\s{8};
   \end{tikzpicture}
   \end{center}

   Wir kreieren einen NDEA f\"ur jeden Knoten, beginnend mit den Bl\"attern.

   % circle definition for enum labels
   \newcommand*\circled[1]{\tikz[baseline=(char.base)]{%
               \node[left=8pt,below=8pt,shape=circle,
         draw,inner sep=1pt,color=blue] (char) {#1};}}
   % set some params for all the pictures in the enumeration
   \tikzset{node distance=3em,auto,
            semithick, every state/.style={minimum size=10pt,inner sep=2pt}}
   \begin{multicols}{2}
   \begin{enumerate}[label=\protect\circled{\arabic*}]
      \item 
         \parbox{\linewidth}{
            \begin{tikzpicture}
               \node[state,initial] (1) {}; 
               \node[state,accepting] (2) [right of=1] {};
               \path[->] (1) edge node {$b$} (2);
            \end{tikzpicture}
         }
      \item
         \parbox{\linewidth}{
            \begin{tikzpicture}
               \node[state,initial] (1) {};
               \node[state,accepting] (2) [right of=1] {};
               \path[->] (1) edge node {$a$} (2);
            \end{tikzpicture}
         }
      \item
         \parbox{\linewidth}{
         \begin{tikzpicture}
            \node[state,initial] (1) {};
            \node[state,accepting] (2) [right of=1] {};
            \path[->] (1) edge node {$d$} (2);
         \end{tikzpicture}
         }
      \item
         \parbox{\linewidth}{
            \begin{tikzpicture}
               \node[state,initial] (1) {};
               \node[state,accepting] (2) [right of=1] {};
               \path[->] (1) edge node {$c$} (2);
            \end{tikzpicture}
         }
   \item
      \parbox{\linewidth}{
         \begin{tikzpicture}
            \node[state,initial] (1) {};
            \node[state] (2) [above right of=1] {};
            \node[state] (3) [below right of=1] {};
            \node[state,accepting] (4) [right of=2] {};
            \node[state,accepting] (5) [right of=3] {};
            \path[->] (1) edge[bend left] node {$\varepsilon$} (2)
                      (1) edge[bend right] node {$\varepsilon$} (3)
                      (2) edge node {$b$} (4)
                      (3) edge node {$a$} (5);
         \end{tikzpicture}
      }
   \item
      \parbox{\linewidth}{
         \begin{tikzpicture}
            \node[state,initial,accepting] (0) {};
            \node[state] (1) [right of=0] {};
            \node[state] (2) [above right of=1] {};
            \node[state] (3) [below right of=1] {};
            \node[state] (4) [above left of=2] {};
            \node[state] (5) [below left of=3] {};
            \path[->] (0) edge node {$\varepsilon$} (1)
                      (1) edge node {$\varepsilon$} (2)
                      (1) edge node {$\varepsilon$} (3)
                      (2) edge node[above] {$b$} (4)
                      (3) edge node {$a$} (5)
                      (4) edge[bend right] node[left] {$\varepsilon$} (0)
                      (5) edge[bend left] node {$\varepsilon$} (0);
         \end{tikzpicture}
      }
   \item
      \parbox{\linewidth}{
         \begin{tikzpicture}
            \node[state,initial] (0) {};
            \node[state] (1) [right of=0]{};
            \node[state] (2) [above right of=1] {};
            \node[state] (3) [below right of=1] {};
            \node[state] (4) [right of=2] {};
            \node[state] (6) [below of=0] {};
            \node[state,accepting] (7) [below of=6] {};
            \node[state] (5) [right of=3] {};
            \path[->] (0) edge node {$\varepsilon$} (1)
                      (1) edge[bend left] node[above=1pt] {$\varepsilon$} (2)
                      (1) edge[bend right] node {$\varepsilon$} (3)
                      (2) edge node {$b$} (4)
                      (0) edge node[left] {$\varepsilon$} (6)
                      (6) edge node {$d$} (7)
                      (4) edge[bend right=120,looseness=1.5] node[above] {$\varepsilon$} (0)
                      (5) edge[bend left=120,looseness=.8,out=60,in=140] node[below] {$\varepsilon$} (0)
                      (3) edge node {$a$} (5);
         \end{tikzpicture}
      }
   \item
      \parbox{\linewidth}{
         \begin{tikzpicture}
            \node[state,initial] (-1) {};
            % subgraph 1
            \node[state] (0) [right of=-1] {};
            \node[state] (1) [right of=0]{};
            \node[state] (2) [above right of=1] {};
            \node[state] (3) [below right of=1] {};
            \node[state] (4) [right of=2] {};
            \node[state] (6) [below of=0] {};
            \node[state,accepting] (7) [below of=6] {};
            \node[state] (5) [right of=3] {};
            \path[->] (0) edge node {$\varepsilon$} (1)
                      (1) edge[bend left] node[above=1pt] {$\varepsilon$} (2)
                      (1) edge[bend right] node {$\varepsilon$} (3)
                      (2) edge node {$b$} (4)
                      (0) edge node[left] {$\varepsilon$} (6)
                      (6) edge node {$d$} (7)
                      (4) edge[bend right=120,looseness=1.5] node[above] {$\varepsilon$} (0)
                      (5) edge[bend left=120,looseness=.8,out=60,in=140] node[below] {$\varepsilon$} (0)
                      (3) edge node {$a$} (5);
           % subgraph 2
           \node[state] (8) [below of=-1] {};
           \node[state,accepting] (9) [below of=8] {};
           \path[->] (-1) edge node {$\varepsilon$}(8)
           (-1) edge node {$\varepsilon$} (0)
            (8) edge node {$c$} (9);
         \end{tikzpicture}
      }

   \end{enumerate}
   \end{multicols}

   \subsection{} 
   
   \def\PuzzleSolutionContent#1{\makebox(1,1){\itshape{#1}}}
   \renewcommand{\PuzzleLineThickness}{1pt}
   \PuzzleSolution
   \begin{Puzzle}{6}{6}
      |b |* |* |* |* |*|.
      |b|c|b|c|a|*|.
      |b|*|d|*|d|*|.
      |b|*|d|*|a|*|.
      |b|b|a|a|d|d|.
      |*|*|*|*|c|*|.
   \end{Puzzle}
   
   \subsection{} 
   
   Wir erstellen f\"ur $k = 0,\ldots,n$ mit $n = 4$ jeweils eine
   \"Ubergangsmatrix $M_k$, wobei $M_k(i,j)$ einen regul\"aren Ausdruck
   enth\"alt, der die Sprache $L_{i,j}^k$ gem\"a\ss{} Vorlesung beschreibt.


   \newcolumntype{M}{>{$}c<{$}}
   \newcommand{\x}{\textcolor{blue}{x}}
   \newcommand{\y}{\textcolor{blue}{y}}
   \newcommand{\z}{\textcolor{blue}{z}}
   \renewcommand{\arraystretch}{1.5}

   Sei $\x{}:=(a\mid b), \y:= aa, \z{}:=ab.$ \\[\baselineskip]

   \begin{multicols}{2}

   \subsubsection{$k = 0$}

   \fbox{
   \begin{tabular}{M|M|M|M|M}
         & 1 & 2 & 3 & 4 \\ \hline
       1 & \varepsilon & b & a & \emptyset \\ \hline
       2 & \emptyset & \varepsilon & \emptyset & \emptyset \\ \hline
      3 & \emptyset & \emptyset & (a \mid b) \mid \varepsilon & c \\ \hline
       4 & a & \emptyset & \emptyset & \varepsilon
   \end{tabular}
   }


   \subsubsection{$ k = 1$}

   \fbox{
   \begin{tabular}{M|M|M|M|M}
         & 1 & 2 & 3 & 4 \\ \hline
       1 & \varepsilon & b & a & \emptyset \\ \hline
       2 & \emptyset & \varepsilon & \emptyset & \emptyset \\ \hline
      3 & \emptyset & \emptyset & \x{} \mid \varepsilon & c \\ \hline
       4 & a & \z{} & \y & \varepsilon
   \end{tabular}
   }

   \subsubsection{$k = 2$}
   
   \fbox{
   \begin{tabular}{M|M|M|M|M}
        & 1 & 2 & 3 & 4 \\ \hline
      1 & \varepsilon & b & a & \emptyset \\ \hline
      2 & \emptyset & \varepsilon & \emptyset & \emptyset \\ \hline
      3 & \emptyset & \emptyset & \x{} \mid \varepsilon & c \\ \hline
      4 & a & \z{} & \y & \varepsilon
   \end{tabular}
   }

   \subsubsection{$k = 3$}
   
   \fbox{
   \begin{tabular}{M|M|M|M|M}
        & 1 & 2 & 3 & 4 \\ \hline
      1 & \varepsilon & b & a\x{}^* & a\x{}^*c \\ \hline
      2 & \emptyset & \varepsilon & \emptyset & \emptyset \\ \hline
      3 & \emptyset & \emptyset & \x{}^* \mid \varepsilon & \x{}^*c\\ \hline
      4 & a & \z{} & \y{}\x{}^* & \varepsilon \mid \y{}\x{}^*c
   \end{tabular}
   }
   \end{multicols}

   \subsubsection{$k = 4$}
   
   \fbox{
   \begin{tabular}{M|M|M|M|M}
        & 1 & \cellcolor{Blue!5} 2 & 3 & \cellcolor{Blue!5} 4 \\ \hline
      \cellcolor{Red!5} 1 & \varepsilon \mid  (a\x{}^*ca)^* & \cellcolor{Blue!5}
       (a\x{}^*ca)^*b & a\x{}^* (c\y{}\x{}^*)^* & \cellcolor{Blue!5}
      a\x{}^*c (\y{}\x{}^*c)^* \\ \hline
      2 & \emptyset & \varepsilon & \emptyset & \emptyset \\ \hline
      3 & \x{}^*ca (a\x{}^*ca)^* & \x{}^* (\x{}^*c\y{})^*c\z{} & \varepsilon \mid
      \x{}^* (c\y{}\x{}^*)^*& \x{}^*c (\y{}\x{}^*c)^*\\ \hline
      4 & a (a\x{}^*ca)^* &  (\y{}\x{}^*c)^*\z{} & \y{}\x{}^* (c\y{}\x{}^*)^* &
      \varepsilon \mid  (\y{}\x{}^*c)^*
   \end{tabular}
   }
   \vspace{\baselineskip}

   Es sind $Z_1$ der einzige Startzustand und $Z_2, Z_4$ Endzust\"ande. Der
   generierte regul\"are Audruck ist daher $(a\x^*ca)^*b\mid a\x^*c(\y\x^*c)^* 
   = (a(a\mid b)^*ca)^*b \mid a(a\mid b)^*c(aa(a\mid b)^*c)^*$
\end{document}
