\documentclass{article}
\title{Informatik \rotatebox[origin=c]{180}{D}\raisebox{2pt}{:} -- Blatt 9}
\author{Rasmus Diederichsen}
\date{\today}
\usepackage[utf8]{inputenc}
\usepackage[ngerman]{babel}
\usepackage[table,svgnames]{xcolor}
\usepackage{wasysym}
\let\iiint\relax % wasysym-amsmath interference
\let\iint\relax
\usepackage{microtype,
   booktabs,
   enumitem,
   multirow,
   listings,
   cwpuzzle,
   tikz,
   multicol,
   IEEEtrantools,
   array,
   amsmath,
   amssymb,
   graphicx, 
lmodern}
\usepackage[pdftitle={Informatik D -- Blatt 9}, 
   pdfauthor={Rasmus Diederichsen}, 
   hyperfootnotes=true,
   colorlinks,
   bookmarksnumbered = true,
   linkcolor = blue,
   plainpages = false,
citecolor = blue]{hyperref}
\usepackage[T1]{fontenc}
\usepackage[all]{hypcap}
\renewcommand{\thesection}{}
\renewcommand{\thesubsection}{Aufgabe \arabic{section}.\arabic{subsection}}
\renewcommand{\thesubsubsection}{}
\setcounter{section}{8}

\begin{document}

\maketitle

\section{} 
\subsection{} 

Das Wort ``nichtselbstidentifizierend'' ist weder selbstidentifizierend noch
nichtselbstidentifizierend. Es kann nicht unter den Begriff
\emph{selbstidentifizierend} fallen, da es nicht mit dem Namen des Begriffs
(``selbstidentifizierend'') übereinstimmt. Es kann nicht
nichtselbstidentifizierend sein, denn dann fiele es unter den Begriff
\emph{nichtselbstidentifizierend}, wäre also zugleich selbstidentifizierend und
nichtselbstidentifizierend. Dies ist ein Widerspruch.

\subsection{} 

Sei das Eingabealphabet von $M$ bezeichnet mit $\Gamma$.
Ein Algorithmus, der für gegebene $M$, $Z_i$ die positive Hälfte entscheidet, ob es eine Eingabe
$w$ gibt, auf der $M$ $Z_i$ erreicht, kann folgendermaßen aussehen:

\lstset{
   basicstyle=\footnotesize\ttfamily,
   language=Python,
   breaklines=true,
   commentstyle=\color{blue},
   keywordstyle=\color{purple}\textbf,
   numberstyle=\tiny\color{gray},
   numbers=left,
   stringstyle=\color{olive},
}

\begin{lstlisting}[escapeinside={*}{*}]
for n in range(0,*$\infty$*):
   for *$w$* in *$\Gamma^{k \le n}$*:
      lasse *$M$* mit Eingabe *$w$* n Schritte ausfuehren
      if *$M$* erreicht *$Z_i$*:
         return 1
\end{lstlisting}

\subsection{} 

\lstset{
   basicstyle=\footnotesize\ttfamily,
   language=C,
   breaklines=true,
   commentstyle=\color{blue},
   keywordstyle=\color{purple}\textbf,
   numberstyle=\tiny\color{gray},
   numbers=left,
   stringstyle=\color{olive},
}
\lstset{
  literate={ö}{{\"o}}1
           {ä}{{\"a}}1
           {ü}{{\"u}}1
           {ß}{{\ss}}1
}

Wir definieren eine Funktion als Programm $\mathcal{P}_1$ mit 
\begin{equation*}
   \mathcal{P}_1(n) =
   \begin{cases}
         1 & \text{falls Eingabe $n$ in einen Zyklus 4, 2, 1 führt} \\
         undef & \text{sonst}
      \end{cases}
   \end{equation*}

   \begin{lstlisting}[escapeinside={*}{*}, title=$\mathcal{P}_1$]
   int *$\mathcal{P}_1$*(int n)
   {
      if (n < 1) return 1 // P sollte für alles anhalten, auch wenn collatz(n) nicht definiert ist
      if (collatz(n) == 4) // wenn Eingabe 4, dann wiederholt sich 4, 2, 1
         return 1
      else return *$\mathcal{P}_1$*(collatz(n))
   }
\end{lstlisting}

Falls $\mathcal{A}\left(\mathcal{P}_1\right) = 1$, so ist die Collatz-Vermutung
bewiesen.

\subsection{} 

Angenommen, es gäbe ein Programm $\mathcal{P}$, das für eine gegebene Turingmaschine
entscheidet, ob sie jemals anhält.
\begin{lstlisting}[language=Java,title=$\mathcal{P}$]
boolean haeltNicht(TuringMachine t)
{
   ...
}
\end{lstlisting}

Wir definieren das Programm $\mathcal{P}_2$
\begin{lstlisting}[language=Java,title=$\mathcal{P}_2$]
void widerspruch(String input)
{
   if(haeltNicht(new TuringMachine(widerspruch))) return;
   else while(true);
}
\end{lstlisting}

Liefert die \texttt{if}-Abfrage \texttt{true}, so hält $\mathcal{P}_2$ für keine
Eingabe an, hält dann aber an. Liefert die Abfrage \texttt{false}, so hält
$\mathcal{P}_2$ für irgendeine Eingabe. Die Eingabe wird aber ignoriert und
$\mathcal{P}_2$ gerät in eine Endlosschleife, hält also für keine Eingabe an.
Dies ist ein Widerspruch. Das Problem ist also nicht entscheidbar.

\subsection{} 

\subsubsection{a)}

\newcommand{\tile}[2]{
      \begin{tabular}{|c|}
         \hline
         \cellcolor{lightgray!10} #1 \\ \hline
         \cellcolor{lightgray!10} #2 \\
         \hline
      \end{tabular}
}

Eine Lösung ist

\vspace{\baselineskip}

\tile{we}{weinh}\tile{in}{ier}\tile{h}{un}\tile{ier}{d}\tile{un}{b}\tile{db}{ier}\tile{ier}{d}\tile{dort}{ort}

\subsubsection{b)}

Es ist keine Lösung möglich. Es gilt $\forall i: |y_i| \ge |x_i| \wedge \sum_i
|x_i| < \sum_i |y_i|$. Es muss jedoch für eine Lösung gelten $\sum_k |x_{i_k}| =
\sum_k |y_{i_k}|$. Dies ist nicht erfüllbar, weil man aufgrund der ersten
Beobachtung ein durch $y_i$s generiertes Längedefizit nicht durch Wiederholung
eines $x_i$s kompensieren kann. Die beiden Zeilen können also niemals die
gleiche Länge haben, es sei denn, man verwendet nur die Steine, die oben wie
unten die gleiche Zeichenzahl besitzen. Dies sind 

\vspace{\baselineskip}
\begin{center}
   \tile{a}{m} und \tile{m}{a}
\end{center}
\vspace{\baselineskip}

Offensichtlich kann man diese beiden auf keine Weise arrangieren, die die
Anforderung erfüllt.

\subsection{} 

\subsubsection{a)}

Die Rückgabe ist immer korrekt, weil $\mathcal{K}$ alle möglichen
\textsc{Purl}-Programme durchprobiert, bis eins die Ausgabe $w$ erzeugt. Da
hierbei von kurz nach lang iteriert wird, kommt hier zudem das kürzestmögliche
heraus. Es gilt $\mathcal{K}(w) \le |w| + \lfloor\log_{10}|w|\rfloor + 2$, da eine
Ausgabe $w$ in jedem Fall durch das Programm $|w|$\texttt{:}$w$ generiert
werden. $\log_{10}|w|$ + 1 liefert gerade die Anzahl von Ziffern der
dezimalkodierten Länge von $w$. Hinzu kommen noch ``\texttt{:}'' sowie alle
Zeichen des Wortes selbst. Mithin ist maximal diese Programmlänge nötig.

\subsubsection{b)}

$\mathcal{K}$ terminiert immer, sofern $|w|<\infty \wedge |\Sigma| < \infty$, da
beide Schleifenindizes dann endliche Wertebereiche haben und ein
\textsc{Purl}-Programm keine Endlosschleifen enthalten kann.

\subsubsection{c)}

Die Nichtberechenbarkeit der Kolmogorov-Komplexität gilt nur für
Turing-vollständige Sprachen. \textsc{Purl} ist dies nicht. Die Sprache erlaubt
keine Endlosschleifen und es gibt keine Variablen oder arithmetische Operationen
und keine Verzweigung. Es ist daher nicht möglich, alle berechenbaren Funktionen
in \textsc{Purl} zu schreiben.  
\end{document}
