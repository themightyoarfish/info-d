\documentclass{article}
\title{Informatik \rotatebox[origin=c]{180}{D}\raisebox{2pt}{:} -- Blatt 11}
\author{Rasmus Diederichsen}
\date{\today}
\usepackage[utf8]{inputenc}
\usepackage[ngerman]{babel}
\usepackage{microtype,
   booktabs,
   enumitem,
   amssymb,
   multirow,
   listings,
   cwpuzzle,
   tikz,
   multicol,
   IEEEtrantools,
   array,
   amsmath,
   amssymb,
   graphicx, 
lmodern}
\usepackage[pdftitle={Informatik D -- Blatt 11}, 
   pdfauthor={Rasmus Diederichsen}, 
   hyperfootnotes=true,
   colorlinks,
   bookmarksnumbered = true,
   linkcolor = blue,
   plainpages = false,
citecolor = blue]{hyperref}
\usepackage[T1]{fontenc}
\usepackage[all]{hypcap}
\renewcommand{\thesection}{}
\renewcommand{\thesubsection}{Aufgabe \arabic{section}.\arabic{subsection}}
\renewcommand{\thesubsubsection}{}
\setcounter{section}{10}

\begin{document}

\maketitle

\section{} 

\subsection{} 

\newcommand{\NP}{N\kern-2ptP}

Der Beweis ist Bullshit. Es wird reduziert von dem in Polynomialzeit lösbaren
Problem \textsc{Eulerkreis} auf das Problem \textsc{Hamiltonkreis}. Dies zeigt
aber nur, das \textsc{Hamiltonkreis} mindestens so schwer ist wie
\textsc{Eulerkreis}, was nicht überrascht, da \textsc{Hamiltonkreis}
$\NP$-vollständig ist. Um zu zeigen, dass
\textsc{Eulerkreis} $\NP$-vollständig ist, müsste man aber zeigen, dass sich
\textsc{Hamiltonkreis} in polynomiell-deterministischer Zeit auf
\textsc{Eulerkreis} reduzieren lässt. Dann wäre bewiesen, dass
\textsc{Eulerkreis} mindestens so schwer ist wie \textsc{Hamiltonkreis}.


\subsection{} 

\subsubsection{a)}

Gegeben einen Zeugen für die Lösbarkeit einer Instanz (also einen Weg, dessen
Länge angeblich $\le k$ ist und der alle Knoten genau einmal enthält), ist es
einfach in polynomieller Zeit, nachzuprüfen ob 
\begin{enumerate}
   \item er wirklich kumulierte Kosten $\le k$ hat (in Linearzeit prüfbar)
   \item er alle Knoten enthält (ebenfalls in Linearzeit prüfbar)
\end{enumerate}

\subsubsection{b)} 

Wie wissen (s.o), dass \textsc{Hamiltonkreis} $\NP$-vollständig ist. Wir müssen
also zeigen, dass sich jede Probleminstanz von \textsc{Hamiltonkreis} auf eine
Instanz von \textsc{TSP} reduzieren lässt. 

Gegeben einen Graphen $G = (V,E)$, $|V| = n$, für den ein Hamilton-Kreis gefunden werden soll,
können wir die Problemstellung umformulieren. Sei $G^\prime = (V^\prime,E^\prime)$ ein vollständiger
Graph mit $V^\prime = V$ und $E^\prime = \{(u,v) \mid u, v \in V^\prime\}$. Es
sind also alle Vertices miteinander verbunden, auch wenn es zwischen ihnen in
$G$ keine Kante gab. Die Kantenkosten seien durch die Funktion
\begin{equation*}
   d(e = (u,v)\in E^\prime) = \begin{cases}
      1 & \text{falls } (u,v) \in E \\
      2 & \text{sonst}
   \end{cases}
\end{equation*}
gegeben. Die Fragestellung lautet nun

\tikzstyle{mybox}=[draw=teal!50, fill=lightgray!5, very thick,
      rectangle, rounded corners, inner sep=10pt]

\begin{center}
   \begin{tikzpicture}
      \node [mybox] (box){
         \begin{minipage}{0.8\textwidth}
            Gibt es einen Weg in $G^\prime$, der jeden Knoten genau einmal besucht und
            dessen kumulierte Kosten ($\le$) $n$ sind?
         \end{minipage}
      };
   \end{tikzpicture}

\end{center}

\subsubsection{c)}

Es bleibt zu zeigen

\begin{center}
   \begin{tikzpicture}
      \node[mybox] (box) {
         \begin{minipage}{.8\textwidth}
            $G$ enthält einen Hamiltonkreis
            \textcolor{teal}{$\Longleftrightarrow$} $G^\prime$ enthält eine
            Rundtour $\le n$.
         \end{minipage}
      };
   \end{tikzpicture}
\end{center}
\begin{itemize}
   \item[``\textcolor{teal}{$\Rightarrow$}'']
      Falls $G$ einen Hamiltonkreis besitzt, dann enthält dieser jeden Knoten
      genau einmal und derselbe Kreis existiert in $G^\prime$. Da in $G^\prime$
      alle Kanten aus $G$ die Kosten 1 besitzen, hat die Rundtour in $G^\prime$
      die Kosten maximal $n$, da maximal $n$ Kanten zwischen $n$ Knoten besucht
      werden können.
   \item[``\textcolor{teal}{$\Leftarrow$}'']
      Wenn es in $G^\prime$ eine Rundtour mit Kosten $\le n$ gibt,
      so sind die kumulierten Kosten die Summe aus $n$ Kantenkosten
      (damit ein Kreis entstehen kann). Es muss also jede Kante die
      Kosten 1 haben, da ansonsten $\sum_{i=1}^n c_i \le n$
      nicht erfüllbar ist mit $c\in\{1,2\}$. Dies wiederum bedeutet
      aber (nach Definition von $d$ oben), dass derselbe Kreis auch
      in $G$ existiert.
\end{itemize}

\subsection{} 

\subsubsection{a)}

\lstset{
   basicstyle=\footnotesize\ttfamily,
   language=Python,
   breaklines=true,
   commentstyle=\color{blue},
   keywordstyle=\color{purple}\textbf,
   numberstyle=\tiny\color{gray},
   numbers=left,
   stringstyle=\color{olive},
}

Das Problem lässt sich in Polynomialzeit per Brute-Force-Search lösen.
\begin{lstlisting}[escapeinside={*}{*}, title=Algorithmus für
\textsc{Clique}-Lösung]
def find_clique(graph, k):
   for g in graph.subgraphs: 
      for (v1, v2) in [(u,v) for u in g.vertices for v in g.vertives if u != v]:
         if !g.hasEdge(v1, v2):
            return false
      return true, g
\end{lstlisting}

\subsubsection{b)}

Es gibt in einem Graphen mit Maximalgrad $k$ und Knotenzahl $n$ nur $n^k$ viele
Teilgraphen, die theoretisch eine Clique bilden können, da diese nur $k$ Knoten
enthalten kann (weil sie vollständig verbunden sein muss). In der Clique muss
jeder der $k$ Knoten $k$ Kanten besitzen und diese müssen ihn mit den anderen
Knoten in der Clique verbinden. Um die Verbundenheit zu prüfen, muss man also
$k^2$ Kanten betrachten. Der Algorithmus läuft also in
$\mathcal{O}\left(n^kk^2\right)$, in diesem Fall $\mathcal{O}\left(n^5\right)$.

Für jede Konstante $k$ läuft der Algorithmus aber immer in Polynomialzeit.

\end{document}
