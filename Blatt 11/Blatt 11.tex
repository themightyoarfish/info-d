\documentclass{article}
\title{Informatik \rotatebox[origin=c]{180}{D}\raisebox{2pt}{:} -- Blatt 11}
\author{Rasmus Diederichsen}
\date{\today}
\usepackage[utf8]{inputenc}
\usepackage[ngerman]{babel}
\usepackage{microtype,
   booktabs,
   enumitem,
   amssymb,
   multirow,
   listings,
   cwpuzzle,
   tikz,
   multicol,
   IEEEtrantools,
   array,
   amsmath,
   amssymb,
   graphicx, 
lmodern}
\usepackage[pdftitle={Informatik D -- Blatt 11}, 
   pdfauthor={Rasmus Diederichsen}, 
   hyperfootnotes=true,
   colorlinks,
   bookmarksnumbered = true,
   linkcolor = blue,
   plainpages = false,
citecolor = blue]{hyperref}
\usepackage[T1]{fontenc}

% % garamond!
% \usepackage{textcomp}
% \renewcommand{\rmdefault}{ugm}
% \usepackage[urw-garamond]{mathdesign}

\usepackage[all]{hypcap}
\renewcommand{\thesection}{}
\renewcommand{\thesubsection}{Aufgabe \arabic{section}.\arabic{subsection}}
\renewcommand{\thesubsubsection}{}
\setcounter{section}{10}


\newenvironment{mybox}[1]{
   \begin{tikzpicture}%
      \node[draw=teal!50, fill=lightgray!5, very thick,%
      rectangle, rounded corners, inner sep=10pt] (box) \bgroup%
         \begin{minipage}{#1}%
         }{%
         \end{minipage}%
      \egroup ;%
   \end{tikzpicture}%
}

\begin{document}

\maketitle

\section{} 

\subsection{} 

\newcommand{\NP}{N\kern-2ptP}

Der Beweis ist Bullshit. Es wird reduziert von dem in Polynomialzeit lösbaren
Problem \textsc{Eulerkreis} auf das Problem \textsc{Hamiltonkreis}. Dies zeigt
aber nur, das \textsc{Hamiltonkreis} mindestens so schwer ist wie
\textsc{Eulerkreis}, was nicht überrascht, da \textsc{Hamiltonkreis}
$\NP$-vollständig ist. Um zu zeigen, dass
\textsc{Eulerkreis} $\NP$-vollständig ist, müsste man aber zeigen, dass sich
\textsc{Hamiltonkreis} in polynomiell-deterministischer Zeit auf
\textsc{Eulerkreis} reduzieren lässt. Dann wäre bewiesen, dass
\textsc{Eulerkreis} mindestens so schwer ist wie \textsc{Hamiltonkreis}.


\subsection{} 

\subsubsection{a)}

Gegeben einen Zeugen für die Lösbarkeit einer Instanz (also einen Weg, dessen
Länge angeblich $\le k$ ist und der alle Knoten genau einmal enthält), ist es
einfach in polynomieller Zeit, nachzuprüfen ob 
\begin{enumerate}
   \item er wirklich kumulierte Kosten $\le k$ hat (in Linearzeit prüfbar)
   \item er alle Knoten enthält (ebenfalls in Linearzeit prüfbar)
\end{enumerate}

\subsubsection{b)} 

Wie wissen (s.o), dass \textsc{Hamiltonkreis} $\NP$-vollständig ist. Wir müssen
also zeigen, dass sich jede Probleminstanz von \textsc{Hamiltonkreis} auf eine
Instanz von \textsc{TSP} reduzieren lässt. 

Gegeben einen Graphen $G = (V,E)$, $|V| = n$, für den ein Hamilton-Kreis gefunden werden soll,
können wir die Problemstellung umformulieren. Sei $G^\prime = (V^\prime,E^\prime)$ ein vollständiger
Graph mit $V^\prime = V$ und $E^\prime = \{(u,v) \mid u, v \in V^\prime, u \neq
v\}$. Es
sind also alle Vertices miteinander verbunden, auch wenn es zwischen ihnen in
$G$ keine Kante gab. Die Kantenkosten seien durch die Funktion
\begin{equation*}
   d(e = (u,v)\in E^\prime) = \begin{cases}
      1 & \text{falls } (u,v) \in E \\
      2 & \text{sonst}
   \end{cases}
\end{equation*}
gegeben. Die Fragestellung lautet nun

\begin{center}
   \begin{mybox}{.8\textwidth}
      Gibt es einen Weg in $G^\prime$, der jeden Knoten genau einmal besucht und
      dessen kumulierte Kosten ($\le$) $n$ sind?
   \end{mybox}

\end{center}

\subsubsection{c)}

Es bleibt zu zeigen

\begin{center}
   \begin{mybox}{.8\textwidth}
      $G$ enthält einen Hamiltonkreis
      \textcolor{teal}{$\Longleftrightarrow$} $G^\prime$ enthält eine
      Rundtour $\le n$.
   \end{mybox}
\end{center}

\begin{itemize}
   \item[``\textcolor{teal}{$\Rightarrow$}'']
      Falls $G$ einen Hamiltonkreis besitzt, dann enthält dieser jeden Knoten
      genau einmal und derselbe Kreis existiert in $G^\prime$. Da in $G^\prime$
      alle Kanten aus $G$ die Kosten 1 besitzen, hat die Rundtour in $G^\prime$
      die Kosten maximal $n$, da maximal $n$ Kanten zwischen $n$ Knoten besucht
      werden können.
   \item[``\textcolor{teal}{$\Leftarrow$}'']
      Wenn es in $G^\prime$ eine Rundtour mit Kosten $\le n$ gibt,
      so sind die kumulierten Kosten die Summe aus $n$ Kantenkosten
      (damit ein Kreis entstehen kann). Es muss also jede Kante die
      Kosten 1 haben, da ansonsten $\sum_{i=1}^n c_i \le n$
      nicht erfüllbar ist mit $c_i\in\{1,2\}$. Dies wiederum bedeutet
      aber (nach Definition von $d$ oben), dass derselbe Kreis auch
      in $G$ existiert.
\end{itemize}
\hfill{}$\Box$

\subsection{} 

\subsubsection{a)}

\lstset{
   basicstyle=\footnotesize\ttfamily,
   language=Python,
   breaklines=true,
   commentstyle=\color{blue},
   keywordstyle=\color{purple}\textbf,
   numberstyle=\tiny\color{gray},
   numbers=left,
   stringstyle=\color{olive},
}

Das Problem lässt sich in Polynomialzeit per Brute-Force-Search lösen.
\begin{lstlisting}[escapeinside={*}{*}, title=Algorithmus für
\textsc{Clique}-Lösung]
def find_clique (graph, k):
   for g in graph.subgraphs(k): # get subgraphs of size <= k
      found = true
      for (v1, v2) in [(u,v) for u in g.vertices for v in g.vertives if u != v]:
         if !g.hasEdge(v1, v2):
            found = false
            break # skip rest of subgraph
      if found:
         return true, g
   return false
\end{lstlisting}

\subsubsection{b)}

Es gibt in einem Graphen mit Maximalgrad $k$ und Knotenzahl $n$ nur
$n^k$ (bzw. $\frac{n!}{(n-k-1)!}$, wenn die Clique maximal sein soll)  viele
Teilgraphen, die theoretisch eine Clique bilden können, da diese nur $k$ Knoten
enthalten kann (weil sie vollständig verbunden sein muss). In der Clique muss
jeder der $k$ Knoten $k$ Kanten besitzen und diese müssen ihn mit den anderen
Knoten in der Clique verbinden. Um die Verbundenheit zu prüfen, muss man also
$k^2$ Kanten betrachten. Der Algorithmus läuft also in
$\mathcal{O}\left(n^k k^2\right)$, in diesem Fall $\mathcal{O}\left(n^5\right)$.

Für jede Konstante $k$ läuft der Algorithmus immer in Polynomialzeit.

\hfill{}$\Box$
\subsection{} 

Wir können zumindest die Vermutung anstellen, dass der Beweis prinzipiell nicht
funktionieren kann. Damit \textsc{NonSat} in $\NP$ ist, muss ein Zeuge für die
Nichterfüllbarkeit einer Formel in polynomiell-deterministischer Zeit
verifizierbar sein. Die Frage ist, wie so ein Zeuge überhaupt aussehen kann.
Meines Erachtens ist die einzige Art von Zeuge, die die Nichterfüllbarkeit
beweist, eine Enumeration sämtlicher möglicher Variablenbelegungen mit dem Wert,
zu dem die Formel mit der Belegung evaluiert (nämlich \texttt{false}). Dieser Zeuge müsste bei $n$
Variablen die Länge $k\cdot2^n$ haben. Ein exponetieller Zeuge ist nicht in
polynomieller Zeit verifizierbar. Das Problem kann also nicht in $\NP$ sein (es
sei denn, es gibt irgendeinen schlaueren Weg, einen Zeugen zu spezifizieren).

\subsection{} 

\subsubsection{a)}

Sei $F = \left\{f_1, \dots, f_m\right\}$ die Menge aller Farben. Seien $H_1,
\dots, H_n \subseteq F$ Mengen mit je höchstens drei Elementen. Gibt es eine
Teilmenge $F_{\text{schön}}\subseteq F$ der Farben, sodass $\forall i \le n:
|F_{\text{schön}} \cap H_i| = 1$?

\subsubsection{b)}

\begin{equation*}
   [[x_i \vee x_j \vee x_k]]
\end{equation*}

\subsubsection{c)}

\textbf{\textsc{Asterix} ist in $\NP$}
\vspace{\baselineskip}

Offensichtlich ist in polynomialzeit feststellbar, ob jede Farbe maximal einmal
pro Haufen vorkommt.

\vspace{\baselineskip}
\noindent\textbf{\textsc{Asterix} ist $\NP$-vollständig}
\vspace{\baselineskip}

Wir reduzieren eine beliebige \textsc{3-SAT}-Instanz $\mathcal{F}$ auf \textsc{Asterix} wie
folgt. 

\begin{enumerate}
   \item Für jedes Literal $x_i$ in $\mathcal{F}$, führe Literal $x_i^\prime$
      ein und ersetze jedes Vorkommen von $\neg x_i$ durch $x_i^\prime$. Erzeuge
      für jede Ersetzung eine Klausel $[[x_i \vee x_i^\prime]]$.
   \item Für jede Klausel in $\mathcal{F}$ mit 3 Literalen $x_i, x_j, x_k$, kreiere eine Klausel 
      \begin{equation*}
         [[x_i^\prime \vee a \vee b]] \wedge [[x_j \vee b \vee c]] \wedge [[x_k^\prime \vee c
         \vee d]],
      \end{equation*} wobei $a, b, c, d$ neue Variablen sind.
   \item Für jede Klausel in $\mathcal{F}$ mit zwei Literalen $x_i, x_j$, führe
      eine neue Variable $e$ ein, sowie $e^\prime := \neg e$. Schreibe die
      Klausel um zu
      \begin{equation*}
         (x_i \vee x_j \vee e) \wedge (x_i \vee x_j \vee e^\prime).
      \end{equation*}
      Diese hat dieselben Wahrheitsbedingungen wie zuvor. Transformiere nun
      beide Konjunkte wie oben zu
      \begin{IEEEeqnarray*}{c}
         [[x_i^\prime \vee a \vee b]] \wedge [[x_j \vee b \vee c]] \wedge [[e^\prime \vee c \vee d]] \\
         \wedge  \\{}
         [[x_i^\prime \vee a \vee b]] \wedge [[x_j \vee b \vee c]] \wedge [[e \vee c \vee d]]
      \end{IEEEeqnarray*}
      wobei $a, b, c, d$ neue Variablen sind. Füge dann noch die Klausel 
      \begin{equation*}
         [[e \vee e^\prime]]
      \end{equation*}
      hinzu.
   \item Für jede Klausel in $\mathcal{F}$ mit einem Literal, kreiere eine
      Klausel
      \begin{equation*}
         [[x_i]].
      \end{equation*}
   \item Verbinde alle Klauseln mit $\wedge$.
\end{enumerate}

Die Umformungen sind allesamt wahrheitserhaltend, daher ist die
\textsc{3-SAT}-Instanz lösbar genau dann wenn, die \textsc{Asterix}-Instanz
lösbar ist.

\hfill{}$\Box$
\subsection{} 
\subsubsection{a)}

\textbf{Definition von SetCover}
\vspace{\baselineskip}

Das \textsc{SetCover}-Problem formuliert als Entscheidungsproblem ist wie folgt
definiert:

\begin{center}
   \begin{mybox}{.8\textwidth}
      Gegeben ein Universum $\mathcal{U} = \left\{1, 2, \dots, m\right\}$, eine Menge
      von Mengen $\mathcal{S} = \left\{S_1, \dots, S_n\right\}$ mit
      $\bigcup_{i=1}^n S_i = \mathcal{U}$ und ein $k \in \mathbb{N}$, gibt es
      eine Teilmenge $\mathcal{C} = \left\{C_1, \dots, C_k\right\} \subseteq
      \mathcal{S}$ mit $\bigcup_{i=1}^k C_i = \mathcal{U}$ und
      $|\mathcal{C}| \le k$?
   \end{mybox}
\end{center}

\textbf{Komplexität von SetCover}
\vspace{\baselineskip}

Wir zeigen, dass \textsc{SetCover} $\NP$-vollständig ist.

\begin{enumerate}
   \item Gegeben ein Zeuge $\mathcal{Z} = \{S_1,\ldots,S_k\}
      \subseteq \mathcal{S}$, für jedes $S_i$ und für jedes $l \in S_i$ entferne $l$
      aus $\mathcal{U}$. Am Ende prüfe, ob $\mathcal{U} = \emptyset$. Da die $n$
      $S_i$ jeweils maximal $m$ Elemente enthalten können und die Suche in
      $\mathcal{U}$ linearen Aufwand bedeutet, sind die Gesamtkosten im Worst
      Case $\mathcal{O}(n\cdot m^2)$, also polynomiell.
   \item Wir zeigen die Härte durch Reduktion von \textsc{VertexCover}. 
      Sei $VC = (G = (V,E), k)$ eine beliebige \textsc{VertexCover}-Instanz.
      Erstelle eine \textsc{SetCover}-Instanz $SC = (\mathcal{U, S}, k)$ mit
      $\mathcal{U} = \{\{u,v\} \mid (u,v) \in E\}$ und $\mathcal{S} = \{S_i \mid
      S_i = \{\{u,v\} \mid V_u = V_i \vee V_v = V_i\}\}$. $\mathcal{S}$ ist also eine
      Menge, die für jeden Knoten die Menge der von ihm ausgehenden Kanten
      enthält. Diese Reduktion ist offensichtlich schlimmstenfalls quadratisch
      in $|V|$.

      Zu zeigen ist:

      \begin{center}
         \begin{mybox}{.8\textwidth}
            \begin{center}
               \textsc{SetCover} ist erfüllbar
               \textcolor{teal}{$\Leftrightarrow$} \textsc{VertexCover}
               erfüllbar.
            \end{center}
         \end{mybox}
      \end{center}

      \begin{itemize}
         \item[``\textcolor{teal}{$\Rightarrow$}'']
            Sei $\mathcal{C}$ eine Lösung der \textsc{SetCover}-Instanz für $k$. Dann
            enthält $\mathcal{C} \subseteq \mathcal{S}$ alle Kanten von $G$, da
            $\mathcal{U}$ alle Kanten enthält und von $\mathcal{C}$ vollständig
            abgedeckt wird.

            Jedes $S_i$ entspricht genau einem $V_i \in V$. Die Menge $\{V_i
            \mid S_i \in \mathcal{C}\}$ ist also ein \textsc{Vertex Cover} für
            $G$ mit $k$ Knoten.
         \item[``\textcolor{teal}{$\Leftarrow$}'']
            Sei $K \subseteq V$ ein \textsc{Vertex Cover} für $G$ mit $|K| =
            k$. Dann ist jede Kante in $G$ inzident zu einem Knoten in $K$.
            Konstruieren wir also für jeden Knoten in $K$ die Menge seiner
            inzidenten Kanten und vereinigen diese Kantenmengen, erhalten wir
            gerade $E$, was äquivalent zu $\mathcal{U}$ ist (s.o.). Die Menge
            $\mathcal{K} = \{S_i \mid S_i = \{\{u,v\} \mid (u,v)\text{ inzident
            zu }v_i \in K\}\}$ ist also gerade ein \textsc{Set Cover} für
            $\mathcal{U}$ mit $|\mathcal{K}| = k$.
      \end{itemize}
\end{enumerate}
\hfill{}$\Box$

\subsubsection{b)}

Das \textsc{Kgb}-Problem lässt sich leicht in ein \textsc{SetCover}-Problem
umformulieren. Es sei
\begin{itemize}[label=$\circ$]
   \item $\mathcal{U}$ die Menge aller Studierenden, nummeriert von 1 bis $n$,
   \item $\mathcal{S} = \{S_i = \{ s \mid \text{ Student/in $s$ hat Zeit an
      Termin $i$}\}\}$ die Menge von Mengen, die jeweils die Studierenden umfassen,
      die sich für einen gegebenen Termin eingetragen haben.
\end{itemize}

Offensichtlich ist das Problem genau dann gelöst, wenn es eine Teilmenge
$\mathcal{C} \subseteq \mathcal{S}$ gibt, sodass alle Studierenden aus
$\mathcal{U}$ abgedeckt sind durch die Auswahl der in $\mathcal{C}$ enthaltenen
Termine und $|\mathcal{C}| \le k$.

\hfill{}$\Box$
\end{document}
