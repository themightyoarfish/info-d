\documentclass{article}
\title{Informatik \rotatebox[origin=c]{180}{D}\raisebox{2pt}{:} -- Blatt 2}
\author{Rasmus Diederichsen}
\date{\today}
\usepackage[ngerman]{babel}
\usepackage{microtype,
            xcolor,
            cwpuzzle,
            tikzsymbols,
            IEEEtrantools,
            array,
            amsmath,
            amssymb,
            graphicx, subcaption,
            lmodern}
\usepackage[pdftitle={Informatik D -- Blatt 2}, 
       pdfauthor={Rasmus Diederichsen}, 
       hyperfootnotes=true,
       colorlinks,
       bookmarksnumbered = true,
       linkcolor = blue,
       plainpages = false,
       citecolor = blue]{hyperref}
\usepackage[T1]{fontenc}
\usepackage[all]{hypcap}
\renewcommand{\thesection}{}
\renewcommand{\thesubsection}{Aufgabe \arabic{section}.\arabic{subsection}}
\renewcommand{\thesubsubsection}{}
\setcounter{section}{1}

\begin{document}

   \maketitle

   \section{} 
   \subsection{} 
   
   Die Regul\"are Grammatik ist

   \begin{IEEEeqnarray*}{rCl}
      S & \rightarrow & 1A \mid 2A \mid 3A \mid 4A \mid 5A \mid 6A \mid 7A \mid 8A \mid 9A \\
        & & \mid 0 \mid 1 \mid 2 \mid 3 \mid 4 \mid 5 \mid 6 \mid 7 \mid 8 \mid 9 \\
      A & \rightarrow & 0A \mid S %1A \mid 2A \mid 3A \mid 4A \mid 5A \mid 6A \mid 7A \mid 8A \mid 9A \\
       % & & \mid 0 \mid 1 \mid 2 \mid 3 \mid 4 \mid 5 \mid 6 \mid 7 \mid 8 \mid 9
   \end{IEEEeqnarray*}

   Ein \"aquivalenter Regul\"arer Ausdruck ist \textcolor{teal}{(0 $\mid$ (1-9)(0-9)\textsuperscript{*})}

   \subsection{} 
   
   \subsubsection{a)}

   Eine M\"oglichkeit ist \textcolor{teal}{$s(a \mid \ldots \mid z)^*(b \mid
   \ldots \mid z)$}.

   \subsubsection{b)}

   Eine M\"oglichkeit ist \textcolor{teal}{$(0 \mid 1)^*00(0 \mid 1)^*$}.
   \subsection{} 
   \usetikzlibrary{arrows,automata}
   
   FSAs f\"ur $\Sigma = \{\Smiley, \Neutrey, \Sadey\}$

   \subsubsection{a)}

   Der Endliche Automat ist

   \begin{center}
      \begin{tikzpicture}[shorten >=1pt,node distance=5em,auto,
         semithick]

         \node[initial,state] (A)                    {A};
         \node[state]         (B) [right of=A] {B};
         \node[state]         (C) [right of=B] {C};
         \node[state]         (D) [right of=C] {D};
         \node[accepting,state] (E) [right of=D]       {E};

         \path[->] (A) edge              node {\Smiley} (B)
         (B) edge node {\Smiley} (C)
         (C) edge[loop above] node {\Smiley,\Neutrey} ()
         (C) edge[bend left] node {\Sadey} (D)
         (D) edge[bend left] node {\Smiley,\Neutrey} (C)
         (D) edge node {\Sadey} (E)
         (E) edge[bend left=80] node {\Smiley,\Neutrey} (C);
      \end{tikzpicture}
   \end{center}

   Ein regul\"arer Ausdruck ist
   \textcolor{teal}{\Smiley\Smiley(\Smiley|\Neutrey|\Sadey)\textsuperscript{*}
   \Sadey\Sadey}

   \subsubsection{b)}

   Der Endliche Automat ist

   \vspace{\baselineskip}
   \begin{center}
      \begin{tikzpicture}[shorten >=1pt,node distance=5em,auto,
         semithick]

         \node[initial,state] (S) {S};
         \node[state] (A) [right of=S] {A};
         \node[state] (B) [right of=A] {B};
         \node[state] (C) [right of=B] {C};
         \node[state] (D) [below of=C] {D};
         \node[state] (E) [left of=D] {E};
         \node[state,accepting] (F) [left of=E] {F};

         \path[->] (S) edge[loop above] node {\Smiley,\Neutrey,\Sadey}()
         (S) edge node {\Neutrey} (A)
         (A) edge node {\Neutrey} (B)
         (B) edge[loop above] node {\Smiley,\Neutrey,\Sadey} ()
         (B) edge node {\Neutrey} (C)
         (C) edge node {\Neutrey} (D)
         (D) edge[loop below] node {\Smiley,\Neutrey,\Sadey} ()
         (D) edge node {\Neutrey} (E)
         (E) edge node {\Neutrey} (F)
         (F) edge[loop below] node {\Smiley,\Neutrey,\Sadey} ();
      \end{tikzpicture}
   \end{center}

   Ein Regul\"arer Ausdruck zu dem Automaten ist \textcolor{teal}{
   (\Smiley|\Neutrey|\Sadey)\textsuperscript{*} \Neutrey\Neutrey
   (\Smiley|\Neutrey|\Sadey)\textsuperscript{*} \Neutrey\Neutrey
   (\Smiley|\Neutrey|\Sadey)\textsuperscript{*} \Neutrey\Neutrey
   (\Smiley|\Neutrey|\Sadey)\textsuperscript{*}}
   
   \subsubsection{c)}

   Der Endliche Automat ist

   \vspace{\baselineskip}
   \begin{center}
      \begin{tikzpicture}[shorten >=1pt,node distance=7em,auto,
         semithick]
         \node[initial,state] (S) {S};
         \node[state] (A) [right of=S] {A};
         \node[state] (B) [right of=A] {B};
         \node[state] (C) [right of=B] {C};
         \node[state] (D) [below of=C] {D};
         \node[accepting,state] (E) [left of=D] {E};
         \node[state,accepting] (F) [left of=E] {F};

         \path[->]
         (S) edge[loop above] node {\Neutrey,\Sadey} ()
         (S) edge[bend left] node {\Smiley} (A)
         (A) edge[bend left] node {\Neutrey,\Sadey} (S)
         (A) edge node {\Smiley} (B)
         (B) edge node {\Neutrey,\Sadey} (C)
         (C) edge[loop right] node {\Neutrey,\Sadey}
         (C) edge[bend left] node {\Smiley} (D)
         (D) edge node {\Smiley} (E)
         (D) edge[bend left] node {\Neutrey,\Sadey} (C)
         (E) edge[bend left] node {\Neutrey,\Sadey} (F)
         (F) edge[loop left] node {\Neutrey,\Sadey} ()
         (F) edge[bend left] node {\Smiley} (E);
      \end{tikzpicture}
   \end{center}

   Ein m\"ogliche \"aquivalenter Regul\"arer Ausdruck ist
   \textcolor{teal}{(\Neutrey{}\textsuperscript{*}\Sadey{}\textsuperscript{*} (\Smiley{} (\Neutrey{}|
      \Sadey{})\textsuperscript{+})\textsuperscript{*})\textsuperscript{*}\Smiley{}\Smiley{} (\Neutrey{}|
   \Sadey{})\textsuperscript{+} (\Smiley{} (\Neutrey{}|
   \Sadey{})\textsuperscript{+})\textsuperscript{*}\Smiley{}\Smiley{} ((\Neutrey{}|
   \Sadey{})\textsuperscript{+} ((\Smiley{} (\Neutrey{}|
\Sadey{})\textsuperscript{+})\textsuperscript{*}|\Smiley{}))\textsuperscript{*}}
 
   \subsection{} 
   
   \begin{center}
      \begin{tikzpicture}[shorten >=1pt,node distance=7em,auto,
         semithick]
         \node[initial,state] (A) {A};
         \node[state] (B) [right of=A]{B};
         \node[state] (C) [right of=B]{C};
         \node[state] (D) [right of=C]{D};
         \node[state] (E) [below of=A] {E};
         \node[state,accepting] (F) [below of=B] {F};
         % \node[state] (J) [right of=C] {J};
         % \node[state] (F) [right of=J] {F};
         % \node[state] (H) [right of=F] {H};
         % \node[state, accepting] (G) [below of=F] {G};

         \path[->]
         (A) edge[bend left] node {0} (B)
         (B) edge[bend left] node {0} (A)
         (B) edge node {1} (C)
         (C) edge[bend left] node {0,1} (D)
         (D) edge[bend left] node {0,1} (C)
         (C) edge node {\#} (F)
         (A) edge[bend left] node {1} (E)
         (E) edge[bend left] node {1} (A)
         (A) edge node[pos=.75] {\#} (F);
         % (A) edge node {0} (B)
         % (A) edge node {1} (C)
         % (B) edge[pos=0.6] node {0} (F)
         % (B) edge node {1} (D)
         % (C) edge[bend right=60] node {1} (F)
         % (D) edge node {\#} (E)
         % (D) edge[bend left] node {0,1} (K)
         % (K) edge[bend left] node {0,1} (D)
         % (F) edge node {\#} (G)
         % (D) edge node {\#} (E)
         % (F) edge[bend left] node {1} (H)
         % (H) edge[bend left] node {1} (F)
         % (J) edge[bend left] node {0} (F)
         % (J) edge[bend left=20,pos=0.25] node {1} (D)
         % (F) edge[bend left] node {0} (J);

      \end{tikzpicture}
   \end{center}

   \subsection{} 
   
   \def\PuzzleSolutionContent#1{\makebox(1,1){\itshape{#1}}}
   \renewcommand{\PuzzleLineThickness}{1pt}
   \PuzzleSolution
   \begin{Puzzle}{6}{6}
      |b |a |b |b |a |b|.
      |b|*|a|b|*|a|.
      |a|b|c|c|b|a|.
      |a|c|b|a|c|c|.
      |a|*|a|c|*|c|.
      |b|c|a|b|c|a|.
   \end{Puzzle}

   \subsection{} 
   
   Eine \"Aquivalente kontextfreie Grammatik ist

   \begin{IEEEeqnarray*}{lCl}
      A & \rightarrow & Ba \mid BCa \mid BCDa \\
      B & \rightarrow & BB \mid Ba \mid bB \mid a \\
      C & \rightarrow & a \mid aa \mid ada \\
      D & \rightarrow & c \mid cC \mid DD
   \end{IEEEeqnarray*}
   
\end{document}
